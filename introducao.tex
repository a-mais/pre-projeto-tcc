\newpage
\section{Introdução}
\justifying

Sistemas de recomendação em aplicativos móveis têm desempenhado um papel cada
vez mais relevante em diferentes setores, especialmente no turismo, onde
proporcionam experiências personalizadas e facilitam a descoberta de serviços e
destinos alinhados aos interesses e necessidades dos usuários. Tais sistemas
utilizam algoritmos inteligentes para analisar dados comportamentais,
preferências explícitas e informações contextuais, sugerindo pontos turísticos,
restaurantes, eventos e outras atrações de acordo com o perfil de cada pessoa.
A crescente popularização de smartphones consolidou os aplicativos de turismo
como ferramentas essenciais, permitindo que recomendações em tempo real sejam
integradas ao planejamento de viagens e roteiros personalizados.

Com o avanço dessas tecnologias, destaca-se a importância de incorporar
critérios de acessibilidade nas recomendações, sobretudo para atender públicos
com necessidades específicas, como idosos e pessoas com deficiência. A
personalização baseada em acessibilidade envolve avaliar obstáculos
arquitetônicos, disponibilidade de serviços especializados, informações sobre o
acesso, e demais fatores que interferem na experiência do turista com
mobilidade reduzida ou outras limitações temporárias ou permanentes. Aplicar
tais critérios em sistemas de recomendação amplia a inclusão, tornando o
turismo mais democrático, seguro e satisfatório para todos.

Em 2023 teve início o projeto de pós-doutorado intitulado ``Uma análise do uso
de IA na geração de rotas turísticas personalizadas para o público idoso: um
estudo de caso no Projeto The Route'', desenvolvido no Instituto Superior de
Engenharia do Porto (ISEP/IPP), em Portugal. O Projeto The Route
~\cite{faria2019theroute} evoluiu para o GrouPlanner~\cite{Alves2022},
coordenado pelo GECAD/ISEP-IPP, incorporando recomendações de roteiros para
grupos baseadas nos traços de personalidade do modelo Big Five~\cite{Costa1992}
e técnicas de negociação para lidar com heterogeneidade grupal na região do
Porto e norte de Portugal~\cite{Alves2023}.

Inspirado nesse contexto e na evolução das pesquisas, estudou-se a viabilidade
de aplicar tais conceitos ao turismo acessível no Brasil, com ênfase na
população idosa. Assim, um projeto de iniciação científica foi desenvolvido no
Instituto Federal do Maranhão (IFMA) Campus São Luís - Monte Castelo entre
agosto de 2024 e agosto de 2025, intitulado ``Um estudo sobre os algoritmos
inteligentes utilizados no Sistema de Recomendação para Grupos aplicado ao
Turismo: um estudo de caso do Projeto GrouPlanner''. O projeto resultou na
implementação de um sistema de recomendação o qual foi nomeado Persona Tour,
baseado no algoritmo d-means \cite{Alves2024}, adaptado para considerar
aspectos de acessibilidade em pontos turísticos na cidade de São Luís - MA. O
sistema utiliza o modelo BFI para agrupar usuários com perfis psicológicos
semelhantes, recomendando pontos turísticos que atendam às suas preferências e
necessidades específicas de acessibilidade. Com base nos resultados positivos
obtidos nesse projeto inicial, o presente trabalho propõe a continuidade e
adaptação desses avanços tecnológicos por meio do desenvolvimento do aplicativo
para o serviço Persona Tour.

O objetivo central consiste em desenvolver uma aplicação móvel voltada ao
turismo acessível que faz uso do sistema de recomendação turística para
obtenção dos pontos turísticos, com especial atenção às necessidades do público
idoso e da pessoa com deficiência. O aplicativo visa oferecer uma experiência
turística personalizada, acessível e sensível às limitações e preferências do
usuário, utilizando o modelo psicológico BFI (Big Five Inventory – BFI) e uma
rede colaborativa dentro do App, reforçando os recursos disponíveis em pontos
turísticos.

\newpage
% ---------------------------------------------------------------
% 1.1 PROBLEMA
% ---------------------------------------------------------------
% \newpage
\subsection{Problema}
\justifying
A acessibilidade é um fator decisivo para a efetiva inclusão no
turismo \cite{UNWTO2016}. Para turistas com deficiência
ou mobilidade reduzida, não basta apenas saber “o que visitar”,
mas se os locais dispõem de recursos como rampas, banheiros
adaptados, piso tátil, sinalização adequada ou rotas sem degraus,
informações que raramente aparecem nos catálogos disponíveis de
forma confiável e comparável. Essa lacuna gera barreiras à autonomia
dos viajantes e reduz o interesse em locais que não evidenciam informações
sobre acessibilidade. Ainda assim, o tema da acessibilidade permanece como um dos
aspectos menos contemplados pela maioria dos sistemas de
recomendação turísticos atuais \cite{santos2019poiaccessibility}.
Nesse sentido, a personalização aliada à acessibilidade
torna-se essencial para melhorar a experiência turística e ampliar a
competitividade econômica do setor \cite{Silva2023}

Portanto, esta pesquisa se propõe a responder à seguinte indagação central:
Como possibiltar que o turista idoso ou com dediciencia tenha acesso a
informação sobre pontos turisitcos da cidade de São Luis - MA que estejam de
acordo com suas preferências e limitações, a fim de promover a devida inclusão
social, cultural e digital?

% ---------------------------------------------------------------
% 1.2 JUSTIFICATIVA
% ---------------------------------------------------------------
\newpage
\subsection{Justificativa}
\justifying
Avançando nesse caminho, a presente proposta insere a acessibilidade como eixo
central da personalização,
transformando informações sobre barreiras e recursos adaptadas para recomendação de
pontos turísticos coesos com o usuário. Assim,
não apenas pontos turísticos são sugeridos, mas condições de acessibilidade do
local e preferências individuais do usuário e do grupo.
Com isso, o projeto justifica-se pela necessidade de superar as limitações dos
recomendadores tradicionais, que priorizam popularidade e cliques,
mas desconsideram variáveis críticas de inclusão \cite{Azambuja2021}.
Social e legalmente, a proposta materializa princípios da Lei Brasileira de
Inclusão ao trazer a acessibilidade para o centro do processo decisório
, academicamente, contribui para a inovação em sistemas de necessidade
social \cite{brasil2014turismoacessivel} \cite{brasil2015lei13146}.

Diante desse contexto, a proposta do Persona Tour justifica-se pela necessidade
de promover a inclusão digital e social da pessoa idosa e da pessoa com
deficiência, no turismo, utilizando a tecnologia como mediadora de experiências
significativas, seguras e adaptadas às suas limitações e interesses. O
aplicativo busca não apenas recomendar destinos e pontos de interesse, mas
também compreender as preferências e necessidades de cada usuário, propondo
lugares compatíveis com suas motivações, nível de atividade e preferências
categóricas. Além disso, a interface do aplicativo foca em ser acessível, com
elementos visuais, tipográficos e interativos adaptados à ergonomia e à
usabilidade de pessoas idosas, contribuindo para a redução das barreiras de
acesso à informação turística.

% ---------------------------------------------------------------
% 1.3 OBJETIVOS
% ---------------------------------------------------------------
\newpage
\subsection{Objetivos}
\subsubsection{Objetivo Geral}
Desenvolver um aplicativo móvel de recomendação de pontos turísticos,
delimitado na recomendação de locais da cidade de São Luís, com foco na
acessibilidade pessoas com deficiência e idosos.

\subsubsection{Objetivos Específicos}
\begin{itemize}
      \item[a.] Investigar técnicas de interface intuitiva para os usuários idosos;
      \item[b.] Desenvolver uma API para integrar o sistema de recomendação ao aplicativo
            móvel;
      \item[c.] Implementar um mínimo produto viável do aplicativo móvel;
      \item[d.] Avaliar a eficácia do aplicativo por meio de testes com diferentes perfis
            de usuários, verificando a precisão das recomendações, a adequação às condições
            de acessibilidade e a percepção de usabilidade
\end{itemize}