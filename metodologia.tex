\section{Metodologia}

\justifying

\subsection{DSR - Design Science Research}

Este trabalho caracteriza-se como uma pesquisa aplicada de natureza
exploratória e experimental, fundamentada nos princípios da Design Science
Research (DSR), metodologia reconhecida por sua capacidade de produzir
artefatos tecnológicos que resolvem problemas concretos enquanto geram
conhecimento científico. A pesquisa foi estruturada em quatro fases
interdependentes e iterativas, alinhadas aos objetivos específicos propostos,
integrando técnicas qualitativas e quantitativas para aplicabilidade prática.

\subsubsection{Fase 1: Investigação de Técnicas de Interface Intuitiva para Usuários Idosos}

A primeira fase concentra-se na compreensão profunda das necessidades e
limitações do público idoso no contexto de interfaces digitais móveis.
Iniciou-se com uma revisão sistemática da literatura abrangendo diretrizes de
acessibilidade para idosos, recomendações de usabilidade específicas para
dispositivos móveis e estudos sobre interação humano-computador com foco nesse
público. Foram consultadas as diretrizes WCAG 2.2 (Web Content Accessibility
Guidelines), adaptadas para o contexto mobile, bem como estudos que investigam
características visuais, cognitivas e motoras associadas ao envelhecimento.

Paralelamente à revisão bibliográfica, serão conduzidas entrevistas
semiestruturadas com potenciais usuários com deficiência e idosos, além de
pessoas que trabalham na área do turismo que atuam com esse público. As
entrevistas seguirão a metodologia de pesquisa participativa, permitindo
identificar preferências de design, barreiras de uso e expectativas
relacionadas a aplicativos turísticos. Os dados coletados deverão ser
organizados através de análise temática, gerando requisitos preliminares sobre
tamanho de fonte, contraste de cores, simplicidade de navegação, feedback
audiovisual e redução de passos para completar tarefas.

\subsubsection{Fase 2: Desenvolvimento da API para Integração com o Sistema de Recomendação}

A segunda fase envolverá o projeto e implementação de uma \textit{Application
    Programming Interface} (API) REST responsável pela comunicação entre o
aplicativo móvel, o sistema de recomendação e o banco de dados.

API de Negócios (Java Spring Boot): Responsável pela lógica de negócio completa do aplicativo, incluindo autenticação de usuários, processamento do questionário BFI, integração com a FastAPI Python, gerenciamento de dados de acessibilidade, sistema de comentários e avaliações colaborativas, e persistência de dados em PostgreSQL. Implementa requisitos rigorosos de segurança através de autenticação JWT, validação de entrada, CORS configurado, e criptografia de dados sensíveis (AES-256 em repouso, TLS 1.3 em trânsito).

A validação das APIs incluirá, testes de segurança para identificar
vulnerabilidades, testes funcionais para garantir integração correta entre as
camadas, e testes de conformidade com LGPD.

\subsubsection{Fase 3: Implementação do Mínimo Produto Viável (MVP) do Aplicativo Móvel}

A terceira fase consistirá no desenvolvimento do MVP do aplicativo Flutter,
seguindo a metodologia ágil de desenvolvimento em ciclos iterativos curtos que
permitem ajustes contínuos baseados em feedback.

As funcionalidades essenciais implementadas no MVP incluirão: (1) tela de
cadastro e autenticação de usuários com validação segura; (2) questionário
digital baseado no modelo BFI para captura de traços de personalidade; (3)
interface para visualização de pontos turísticos recomendados com indicadores
de acessibilidade (presença de rampas, elevadores, banheiros adaptados,
sinalização tátil); (4) link para mapa interativo mostrando localização dos
pontos; (5) sistema de filtros considerando preferências pessoais e
necessidades de acessibilidade; (6) criação de listas de pontos turísticos
compartilháveis entre usuários; (7) sistema de avaliação colaborativo onde
usuários relatam experiências de acessibilidade e comentam sobre pontos
turísticos.

A interface será projetada considerando os requisitos levantados na Fase 1,
implementando elementos como botões grandes com áreas de toque ampliadas,
fontes ajustáveis (mínimo 16pt), alto contraste entre texto e fundo (4,5:1),
navegação simplificada com poucos níveis hierárquicos, feedback claro para cada
ação do usuário, e suporte a recursos nativos de acessibilidade (VoiceOver no
iOS e TalkBack no Android). A prototipação será realizada utilizando
ferramentas de design como Figma, permitindo validação visual antes da
codificação.

\subsubsection{Fase 4: Avaliação da Eficácia do Aplicativo através de Testes com Usuários}

A fase final foca na validação do aplicativo através de testes de usabilidade e
aceitação com diferentes perfis de usuários. A amostra será selecionada por
conveniência, incluindo: pessoas idosas com diferentes níveis de familiaridade
com tecnologia; pessoas com deficiência física ou mobilidade reduzida; e
profissionais de turismo e acessibilidade que atuam como validadores técnicos.

Os testes serão realizados em ambiente controlado permitindo observação direta
da interação dos usuários com o aplicativo. Tarefas pré-definidas incluem:
realizar cadastro, responder ao questionário de personalidade, visualizar
recomendações, filtrar pontos por critérios de acessibilidade, e observar
comentários de outros usuários. Os resultados serão registrados para análise
posterior da eficácia do sistema.

A avaliação da precisão das recomendações envolverá análise comparativa entre
as sugestões geradas pelo sistema e as preferências reais declaradas pelos
participantes em questionário pós-teste, calculando métricas de acurácia e
relevância. A adequação às condições de acessibilidade será verificada
confrontando as informações apresentadas pelo aplicativo com dados reais dos
estabelecimentos turísticos.

Dados quantitativos serão analisados através de estatística descritiva
utilizando ferramentas como R ou Python. Dados qualitativos (transcrições de
entrevistas, observações dos testes de usabilidade) serão submetidos à análise
de conteúdo temática, identificando padrões recorrentes, dificuldades comuns e
sugestões de melhorias, atendendo simultaneamente aos requisitos técnicos de
integração com sistemas de recomendação inteligentes e às necessidades reais de
acessibilidade e usabilidade do público-alvo.

\begin{figure}[H]
    \centering
    \IfFileExists{./figuras/fluxograma.jpg}{%
        \includegraphics[width=1\textwidth]{./figuras/fluxograma.jpg}%
    }{%
        \fbox{\begin{minipage}[c][0.3\textheight][c]{1\textwidth}\centering Arquivo de figura ausente: figuras/arquitetura.pdf \end{minipage}}%
    }
    \caption{Figura que ilustra o fluxograma das fases da metodologia. Fonte: Autor.}
\end{figure}