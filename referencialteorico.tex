\section{Referencial Teórico}
\justifying
\subsection{Turismo Acessível e Acessibilidade}

O turismo representa uma das atividades econômicas e sociais mais importantes
para países em desenvolvimento, funcionando como gerador de renda, emprego e
intercâmbio cultural. Contudo, a participação igualitária nessa prática depende
fundamentalmente da existência de infraestrutura e serviços adequados que
considerem as necessidades específicas de pessoas com deficiência e mobilidade
reduzida, incluindo idosos. A acessibilidade turística transcende a
disponibilidade de rampas e banheiros adaptados; ela envolve um conceito mais
amplo que contempla elementos informativos, comunicacionais, tecnológicos e
atitudinais.

Segundo dados recentes do Ministério do Turismo, mais de 53,5\% dos turistas
com deficiência deixaram de viajar para algum destino no Brasil por falta de
acessibilidade, revelando uma exclusão sistemática desse
público~\cite{MTur2024}. O Brasil, embora reconhecido internacionalmente por
suas belezas naturais, apresenta graves deficiências na operacionalização de
políticas de turismo inclusivo, conforme demonstrado em análises de políticas
públicas dos últimos 20 anos~\cite{Vilela2022}.

Um caso emblemático dessa falha institucional é o ``Programa Turismo
Acessível'', projeto desenvolvido pelo próprio governo brasileiro para mapear
acessibilidade em pontos turísticos. O aplicativo, embora bem-intencionado,
apresenta limitações críticas: dados desatualizados, informações inconsistentes
sobre acessibilidade real dos espaços, ausência de validação contínua pelas
comunidades de pessoas com deficiência, e interface pouco acessível para o
público idoso. Muitos pontos turísticos cadastrados como acessíveis carecem de
infraestrutura essencial como rampas inadequadas, banheiros sem adaptações e
falta de sinalização tátil, demonstrando que a presença de informação não
garante qualidade de implementação~\cite{turismoacessivel2025}.

A Lei Brasileira de Inclusão da Pessoa com Deficiência (Lei 13.146/2015) e o
Manual de Turismo Acessível da UNWTO estabelecem que acessibilidade é direito
fundamental, não benefício~\cite{brasil2014turismoacessivel,UNWTO2016}.
Entretanto, a implementação permanece lenta e desigual. O conceito de turismo
acessível deve ser compreendido em múltiplas dimensões complementares. A
acessibilidade arquitetônica refere-se à remoção de barreiras físicas,
incluindo rampas adequadas com inclinação máxima de 8,33\%, corrimões,
elevadores, pisos táteis e banheiros com espaço de manobra mínimo de 1,5m de
diâmetro. A acessibilidade comunicacional envolve a disponibilidade de
informações em formatos alternativos como Braille, áudio, linguagem simples e
descrição de imagens, aliada a sinalização clara e legível com descrição de
ambientes. A acessibilidade atitudinal compreende o treinamento de
profissionais de turismo para atender com qualidade e respeito, promovendo a
compreensão da neurodiversidade e das diferentes formas de deficiência,
valorizando a autonomia do visitante. Por fim, a acessibilidade tecnológica
garante a disponibilidade de recursos assistivos como leitores de tela,
amplificadores e sistemas de amplificação de som, além da integração com
tecnologias móveis e interfaces acessíveis em plataformas digitais.

\subsection{Sistemas de Recomendação no Turismo}

Os Sistemas de Recomendação (SR) têm revolucionado a forma como os indivíduos
descobrem produtos, serviços e informações em ambientes digitais. No contexto
turístico, eles auxiliam viajantes a identificar pontos de interesse relevantes
em meio à sobrecarga informacional, reduzindo significativamente o tempo de
pesquisa e aumentando a qualidade da experiência~\cite{Resnick1997,
    Santos2023}. Um sistema de recomendação funciona como intermediário entre a
diversidade de ofertas turísticas e as preferências específicas de cada
usuário, processando dados comportamentais, demográficos e psicológicos para
gerar sugestões personalizadas.

As abordagens tradicionais em Sistemas de Recomendação Turística baseiam-se
predominantemente em duas técnicas: a filtragem colaborativa, que identifica
usuários com preferências similares e recomenda itens apreciados por esses
``vizinhos''; e a filtragem baseada em conteúdo, que analisa características
dos pontos turísticos como categoria, localização e características,
comparando-as com o perfil do usuário~\cite{Resnick1997}. Contudo, essas
abordagens possuem limitações reconhecidas. A filtragem colaborativa sofre com
o problema de ``partida a frio'' caracterizado por poucos dados históricos de
usuários novos, enquanto a filtragem por conteúdo tende a criar ``bolhas de
informação'' que reduzem a serendipidade e a descoberta de novas experiências.

O \textbf{GrouPlanner}~\cite{Alves2022, Alves2023}, desenvolvido pelo
GECAD/ISEP-IPP (Grupo de Pesquisa em Engenharia Computacional de Sistemas
Adaptativos Complexos do Instituto Superior de Engenharia do Porto), representa
um avanço significativo ao incorporar modelagem de personalidade como base para
recomendação. O GrouPlanner utiliza o modelo psicológico \textbf{Big Five
    Inventory (BFI)}~\cite{Costa1992} para capturar os cinco principais traços de
personalidade humana. A abertura para experiências reflete a propensão para
inovação, criatividade e flexibilidade, influenciando a busca por atrações
inovadoras, culturais e desafiadoras. A conscienciosidade relaciona-se com
organização, planejamento e atenção a detalhes, influenciando a preferência por
roteiros estruturados e segurança. A extroversão determina a sociabilidade e
busca por estimulação, afetando a preferência por atividades em grupo, vida
noturna ou contemplação solitária. A agradabilidade representa a orientação
para cooperação e harmonia, afetando a disposição para turismo comunitário e
experiências inclusivas. Por fim, o neuroticismo refere-se à tendência para
emoções negativas, relacionando-se com sensibilidade a stress ambiental,
necessidade de conforto e segurança.

O BFI é implementado através de um questionário validado
cientificamente~\cite{Costa1992, Andrade2023} que permite classificar usuários
em um espaço multidimensional de personalidade. A principal inovação do
GrouPlanner reside na aplicação do algoritmo \textit{d-means}~\cite{Alves2024}
para agrupar usuários com perfis psicológicos similares e gerar recomendações
coletivas em contextos de grupo, negociando preferências heterogêneas de forma
harmoniosa.

Para o Persona Tour, a arquitetura de recomendação foi adaptada do GrouPlanner,
incorporando uma dimensão crítica ausente em sistemas tradicionais: a modelagem
de restrições de acessibilidade. O sistema processa não apenas traços de
personalidade, mas também informações sobre o tipo e grau de deficiência ou
mobilidade reduzida, necessidades específicas como presença de elevador, piso
tátil, intérprete de Libras e descrição de ambientes, além de preferências de
tipo de turismo ajustadas às capacidades do usuário. A integração entre BFI e
restrições de acessibilidade cria um modelo de recomendação verdadeiramente
inclusivo, que não apenas respeita preferências de personalidade mas também
garante a viabilidade física e segurança da experiência turística. Esse
diferencial posiciona o Persona Tour como inovação no campo do turismo
inteligente (Tourism 4.0), alinhado com políticas públicas de inclusão
social~\cite{Silva2023}.

\subsection{Aplicações Mobile e Experiência do Usuário (UX)}

A proliferação de dispositivos móveis transformou a forma como as pessoas
interagem com informações, serviços e experiências. Aplicativos móveis
representam 90\% do tempo de engajamento digital, tornando-os o canal
preferencial para acesso a serviços turísticos. Contudo, a qualidade e
acessibilidade dessas aplicações variam enormemente, com muitas falhando em
atender necessidades de usuários com deficiência ou idosos~\cite{Bevan2021,
    Bruffaerts2021}.

\subsubsection{Acessibilidade em Aplicações Móveis: WCAG 2.1}

As \textbf{Web Content Accessibility Guidelines 2.1 (WCAG 2.1)}~\cite{W3C2021},
desenvolvidas pelo World Wide Web Consortium, estabelecem padrões
internacionais reconhecidos para acessibilidade digital. Embora originalmente
focadas em conteúdo web, suas diretrizes são amplamente aplicáveis a aplicações
móveis, especialmente em nível AA que garante acessibilidade para 80\% das
necessidades de pessoas com deficiência~\cite{W3C2021, MDPI2021}.

Os princípios WCAG 2.1 fundamentais para aplicações móveis dividem-se em quatro
categorias. A perceptibilidade exige que a informação seja apresentada de forma
compreensível, especialmente através de contraste de cores com razão mínima de
4,5:1 entre texto e fundo, ou 3:1 para texto grande com tamanho igual ou
superior a 18 pontos. Aplicações frequentemente violam esse requisito, causando
ilegibilidade para pessoas com baixa visão. Adicionalmente, o tamanho de texto
deve ser ajustável e a interface deve permanecer utilizável em escalas até
200\%, com descrição de elementos visuais como botões, ícones e imagens que
devem possuir rótulos textuais ou descrições acessíveis para leitores de tela.

A operabilidade garante que todos os componentes sejam operáveis sem exclusão
de usuários com diferentes capacidades motoras ou cognitivas. O tamanho de alvo
tátil é crítico, exigindo que botões e elementos interativos possuam área
mínima de 48x48 pixels lógicos para facilitar ativação por usuários com
limitações motoras finas~\cite{Flutter2025, Dev2023}. Toda funcionalidade deve
ser acessível via teclado, suportando navegação com abas e foco claramente
visível. Leitores de tela como TalkBack para Android e VoiceOver para iOS devem
descrever a interface de forma coerente e permitir navegação completa. Além
disso, mudanças de contexto automáticas devem ser evitadas, pois alterações não
devem ocorrer sem confirmação do usuário, especialmente durante entrada de
dados.

A compreensibilidade exige que interface, comportamento e mensagens sejam
claros e antecipáveis. Isso envolve o uso de linguagem simples com texto
legível e vocabulário apropriado, sendo especialmente importante para usuários
idosos e com deficiência cognitiva~\cite{Recomendacoes2015}. As ações do
usuário devem gerar feedback imediato através de feedback visual, auditivo ou
tátil confirmando sucesso ou indicando erro com sugestão de correção. Padrões,
cores e posicionamento devem permanecer consistentes ao longo da aplicação para
facilitar a navegação intuitiva.

A robustez garante que o aplicativo funcione com diversas tecnologias
assistivas e em diferentes contextos. A compatibilidade com tecnologias
assistivas é fundamental, exigindo funcionamento adequado com leitores de tela,
amplificadores de som e interfaces adaptadas. Comportamento consistente deve
ser garantido em smartphones, tablets e versões antigas e novas do sistema
operacional através de testes em múltiplos dispositivos.

\subsubsection{Recomendações Específicas para Idosos}

Estudos sobre interação de idosos com aplicações
móveis~~\cite{Recomendacoes2015, LUSUI2021, Vilela2022} identificam desafios
cognitivos, visuais e motores específicos que exigem diretrizes adaptadas. A
degradação visual natural caracteriza-se pela redução de acuidade visual,
dificuldade com contraste e aumento de sensibilidade a cintilação,
recomendando-se font-size mínimo de 16-18 pontos, espaçamento entre elementos
ampliado e aumento de espaçamento de linha para melhor legibilidade.

A redução de destreza motora associada ao tremor, artrite e redução de precisão
manual exige que botões sejam significativamente maiores que 48x48 pixels, com
recomendação de 60-72 pixels, mantendo espaçamento amplo entre
elementos~\cite{Recomendacoes2015}. A mudança cognitiva caracteriza-se pelo
aumento de tempo de processamento, dificuldade com abstrações visuais e
tendência à familiaridade com padrões conhecidos. Nesses casos, navegação deve
ser simplificada com máximo de 3 níveis hierárquicos, nomes de botões claros e
descritivos, evitando ícones sem rótulo que possam causar confusão.

A proficiência tecnológica variável entre idosos, onde muitos possuem
experiência limitada com aplicativos móveis, exige que se priorize
familiaridade através de padrões conhecidos como enviar/receber para
comunicação, tutoriais integrados e suporte à ajuda contextual. O design deve
ser construído com a premissa de que o usuário idoso necessita de uma curva de
aprendizado reduzida e de máxima clareza em cada interação.

\subsubsection{Experiência do Usuário (UX) Centrada no Usuário Específico}

A Experiência do Usuário vai além da mera funcionalidade; refere-se ao conjunto
de emoções, atitudes e comportamentos associados ao uso de um
sistema~\cite{Nielsen2012}. Para públicos com necessidades especiais, UX deve
ser construída através de pesquisa participativa, onde o usuário é co-designer
do sistema.

O \textbf{Design Thinking} propõe que aplicações sejam desenvolvidas em
iterações sucessivas. A fase de descoberta busca entender profundamente as
necessidades reais dos usuários. A definição reformula o problema com base em
insights gerados. A ideação gera múltiplas soluções para o problema
identificado. A prototipação constrói versões tangíveis que permitem visualizar
a solução. O teste valida com usuários reais se a solução atende às
necessidades.

Para aplicações turísticas, isso significa envolver desde o início pessoas
idosas e com deficiência em sessões de prototipagem e teste de usabilidade,
capturando não apenas erros funcionais, mas também a satisfação emocional e o
senso de autonomia gerado pela experiência. A participação ativa desses
usuários garante que a aplicação seja realmente acessível e útil, não apenas em
conformidade técnica com diretrizes, mas em experiência prática e
significativa.

\subsection{Tecnologias e Arquitetura do Sistema}

O Projeto é fundamentado em uma arquitetura de quatro camadas que separa
responsabilidades e permite escalabilidade. Essa separação arquitetônica
possibilita que cada camada seja desenvolvida, mantida e evoluída
independentemente, facilitando a incorporação de novas tecnologias, algoritmos
e funcionalidades sem impactar as demais camadas.

\subsubsection{Camada 1: API do Sistema de Recomendação}

A primeira camada implementa o sistema central de recomendação, desenvolvida em
FastAPI Python. Essa API é responsável por processar perfis de personalidade do
usuário e gerar recomendações de pontos turísticos através do algoritmo
d-means, consumindo dados da Google Maps API como terceiro para enriquecer as
recomendações com informações geográficas. O algoritmo d-means recebe como
entrada o perfil BFI do usuário representado por cinco dimensões de
personalidade e suas restrições específicas de acessibilidade, agrupando pontos
turísticos que combinam características que atraem usuários com aquele perfil
psicológico enquanto garantem conformidade com requisitos específicos de
acessibilidade. A integração com Google Maps API permite consumir dados
geográficos, localização, categoria, avaliações, horários e contato dos pontos
turísticos, enriquecendo essas informações com metadados de acessibilidade
provenientes do banco de dados do sistema. O processamento inteligente aplica
lógica sofisticada de matching entre perfil de personalidade e características
dos pontos turísticos, retornando lista ordenada de recomendações com score de
relevância que reflete o grau de correspondência. A validação de dados de
acessibilidade garante que as recomendações atendem rigorosamente aos
requisitos específicos de mobilidade, visão e audição declarados pelo usuário,
evitando sugerir locais que, embora alinhados com a personalidade, possam ser
inacessíveis. Essa camada funciona como motor de inteligência, independente de
tecnologias de frontend ou regras de negócio específicas do aplicativo, podendo
ser facilmente integrada em diferentes contextos.

\subsubsection{Camada 2: API de Negócios e Gestão de Dados}

A segunda camada implementa toda a lógica de negócio do aplicativo,
desenvolvida em Java Spring Boot. Essa API centraliza múltiplas
responsabilidades críticas para o funcionamento da plataforma. O módulo de
autenticação e perfil gerencia cadastro de usuários com armazenamento seguro de
credenciais através de criptografia avançada, validação de email para
verificação de contas e persistência de dados de perfil no banco de dados. O
módulo de questionário BFI implementa o Big Five Inventory validado em
português brasileiro, composto por quinze itens na versão reduzida, coletando
respostas em escala Likert de cinco pontos e calculando automaticamente scores
para cada dimensão de personalidade. O módulo de integração com sistema de
recomendação orquestra a comunicação entre os componentes da plataforma,
capturando o perfil BFI calculado do usuário, coletando suas restrições
específicas de acessibilidade, enviando requisição devidamente formatada à
FastAPI Python para processar recomendações, recebendo a lista de pontos
turísticos recomendados e enriquecendo a resposta com dados complementares
armazenados no banco local, como avaliações anteriores e comentários de outros
usuários.

O módulo de dados de acessibilidade armazena e gerencia metadados detalhados
sobre cada ponto turístico incluindo informações sobre rampas, elevadores,
banheiros adaptados, sinalização tátil e outros requisitos de acessibilidade,
permitindo buscas facetadas por tipo de acessibilidade que o usuário necessita.
O módulo de comentários e avaliações implementa um sistema colaborativo onde
usuários avaliam pontos turísticos, relatam experiências de acessibilidade,
compartilham fotos e comentários que enriquecem a informação disponível para
futuros visitantes. Esse sistema implementa validação por pares e moderação de
conteúdo inadequado para garantir qualidade e confiabilidade das informações. O
módulo de filtros e preferências gerencia preferências pessoais de categorias
turísticas, necessidades específicas de acessibilidade, histórico de visitas e
listas compartilháveis entre usuários, permitindo customização profunda da
experiência. O banco de dados PostgreSQL organiza tabelas estruturadas de
usuários, perfis BFI, restrições de acessibilidade, histórico de recomendações
e interações, sistema de avaliações e comentários colaborativos, além de cache
de dados de pontos turísticos sincronizados com informações da FastAPI Python.

Os requisitos de segurança são rigorosos, implementando autenticação JWT (JSON
Web Tokens) para validação de requisições, validação rigorosa de entrada para
prevenir ataques de injeção, CORS configurado para permitir apenas origens
autorizadas, criptografia de dados sensíveis em repouso utilizando AES-256, TLS
1.3 para comunicação criptografada em trânsito e conformidade total com LGPD
(Lei Geral de Proteção de Dados) através de auditoria contínua.

\subsubsection{Camada 3: Banco de Dados}

A terceira camada é responsável pelo armazenamento persistente e seguro de
todos os dados da plataforma, implementada em PostgreSQL, um banco de dados
relacional robusto, altamente confiável e otimizado para aplicações críticas.
A escolha do PostgreSQL justifica-se por sua capacidade de lidar com volumes
grandes de dados, sua conformidade com padrões SQL e suporte nativo a tipos de
dados complexos como JSON, arrays e geoespaciais, essenciais para a aplicação
turística. O esquema do banco de dados organiza-se em múltiplas tabelas inter-relacionadas
que refletem a estrutura de domínio do sistema~\cite{postgresql2024official}.

\subsubsection{Camada 4: Aplicação Móvel}

A quarta camada é a interface de usuário, implementada em Flutter, framework
open-source de desenvolvimento cross-platform da Google que permite criar
aplicações nativas em iOS e Android com código base único. A escolha do Flutter
justifica-se por múltiplas razões alinhadas aos objetivos de acessibilidade do
projeto. O framework possui suporte nativo a acessibilidade através de
integração built-in com tecnologias assistivas como TalkBack para Android e
VoiceOver para iOS, com widgets que fornecem propriedades como semanticLabel,
onTap e onLongPress para navegação fluida por leitor de tela. Os widgets do
Flutter são acessíveis por padrão, com componentes como Button, TextField e
AppBar já implementando acessibilidade mínima que desenvolvedores podem
estender. O framework oferece suporte a verificação automática de conformidade
com diretrizes WCAG 2.1 através de pacotes especializados. Flutter compila para
código nativo, oferecendo performance superior crucial para dispositivos com
recursos limitados comuns entre usuários idosos. O Hot Reload permite iteração
rápida durante desenvolvimento e teste de acessibilidade, acelerando o ciclo de
desenvolvimento e validação.

Os componentes principais da aplicação Flutter organizam-se em telas
especializadas. A tela de autenticação implementa login e registro com campos
amplos mantendo área mínima de 48 pixels, alta relação de contraste de 4,5:1,
feedback claro de erros indicando exatamente o que corrigir e suporte completo
a navegação por teclado. A tela de questionário BFI distribui a interface em
múltiplas páginas para evitar sobrecarga cognitiva do usuário, exibe indicador
de progresso claro mostrando quantas questões foram respondidas, oferece opção
de pausar e retomar posteriormente, e apresenta escala Likert com rótulos
textuais claros variando de ``Discordo Fortemente'' a ``Concordo Fortemente''.
A tela de recomendações exibe lista de pontos turísticos em card layout
acessível, com cada card mostrando foto com descrição alternativa para leitores
de tela, nome do local, categoria, distância até o ponto e indicadores visuais
de acessibilidade como ``Rampa disponível'', ``Banheiro adaptado'' e
``Estacionamento próximo''.

A tela de detalhes do ponto turístico fornece informação expandida incluindo
endereço completo, mapa interativo para visualizar localização, descrição
detalhada do espaço com audiodescrição de fotos disponível, informações
estruturadas de acessibilidade, contato do estabelecimento, horários de
funcionamento, sugestões de transporte acessível para chegar ao local,
avaliações de outros usuários sobre a acessibilidade e galeria de fotos. A tela
de avaliação e comentários permite que usuários avaliem a acessibilidade real
do ponto turístico, compartilhem experiências pessoais, relatem discrepâncias
entre informações cadastradas e realidade observada e participem de validação
colaborativa de informações. O perfil do usuário permite gerenciar informações
pessoais, restrições de acessibilidade, histórico de visitas anteriores, pontos
marcados como favoritos e configurações avançadas de acessibilidade incluindo
tamanho de fonte ajustável, modo de alto contraste e velocidade de animação.

A implementação de acessibilidade em Flutter segue rigorosamente as diretrizes
WCAG 2.1. A tipografia utiliza font-size mínimo de 16 pontos para corpo de
texto, 18-20 pontos para cabeçalhos, fonte sans-serif como Roboto ou Open Sans
para melhor legibilidade e suporte a aumento até 200\% para usuários com baixa
visão. Cores e contraste mantêm paleta com contraste mínimo de 4,5:1, modo de
alto contraste disponível como opção, e evita-se usar cor como único indicador
de informação, complementando com ícones, texto e padrões visuais. A navegação
estrutura-se hierarquicamente com máximo de três níveis para evitar confusão
cognitiva, botões ``Voltar'' conspícuos em cada página, breadcrumbs em páginas
profundas indicando o caminho percorrido e suporte completo a navegação por
teclado com ordem de foco lógica. Leitores de tela recebem suporte máximo com
todos os elementos interativos possuindo semanticLabel descritivo, imagens
tendo rótulos explicativos, ordem semântica otimizada para fluxo natural e
testes extensivos com TalkBack e VoiceOver.

Interações geram feedback multissensorial através de mudança visual de cor ou
ícone, feedback auditivo com som ou vibração do dispositivo e feedback textual
através de toast ou snackbar explicando resultado da ação. Campos obrigatórios
são claramente marcados com indicadores visuais, erros de validação aparecem
imediatamente com sugestão concreta de correção e ações importantes são
reversíveis, permitindo que usuários desfaçam operações sem medo de
consequências irreversíveis.

\subsubsection{Integração das Quatro Camadas}

\begin{figure}[H]
    \centering
    \IfFileExists{./figuras/arquitetura.pdf}{%
        \includegraphics[width=0.85\textwidth]{./figuras/arquitetura.pdf}%
    }{%
        \fbox{\begin{minipage}[c][0.3\textheight][c]{0.85\textwidth}\centering Arquivo de figura ausente: figuras/arquitetura.pdf \end{minipage}}%
    }
    \caption{Figura que ilustra o fluxo e arquitetura do sistema. Fonte: Autor.}
\end{figure}

O fluxo de dados entre as camadas segue um padrão bem definido que garante
integração \textit{seamless} e responsabilidade clara de cada componente. O
Flutter cliente autentica o usuário e captura seu perfil BFI através do
questionário interativo. Os dados são enviados para a API de Negócios em Java
Spring Boot que processa o BFI, formata a requisição com os parâmetros
necessários e a envia à FastAPI Python. A FastAPI Python executa o algoritmo
\textit{d-means}, consulta Google Maps API para enriquecer dados geográficos,
e retorna lista de recomendações com scores de relevância. Retornando à Java
Spring Boot, enriquece a resposta com comentários, avaliações e informações
complementares do PostgreSQL respondidas por usuários anteriormente, agregando
conhecimento coletivo sobre acessibilidade. Finalmente, Flutter exibe as
recomendações personalizadas de forma visual e acessível com informações claras
de acessibilidade para cada ponto sugerido.