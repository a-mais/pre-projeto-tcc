\thispagestyle{empty}
\section*{RESUMO}
\begin{singlespace} % espaçamento simples conforme NBR 6028
    \justifying
    Este trabalho apresenta o desenvolvimento de um aplicativo mobile de recomendação de pontos
    turísticos da cidade de São Luís com foco em acessibilidade para pessoas idosas e com deficiência. O objetivo é oferecer recomendações
    personalizadas adaptadas às necessidades específicas
    dos usuários, promovendo inclusão social e digital no contexto turístico. A justificativa
    fundamenta-se no dado alarmante de que 53,5\% dos turistas com deficiência deixam de
    viajar no Brasil por falta de acessibilidade, e na ausência de soluções que integrem
    personalidade com acessibilidade como elemento central. O diferencial da proposta
    reside na integração simultânea do modelo psicológico Big Five Inventory (BFI)
    com restrições de acessibilidade, gerando recomendações que respeitam tanto
    preferências individuais quanto garantem viabilidade física e segurança da experiência.
    A metodologia segue Design Science Research com quatro fases: investigação de técnicas
    de interface para idosos via revisão sistemática e pesquisa participativa;
    desenvolvimento de uma API REST em Java Spring Boot para lógica de negócios do
    aplicativo e integração com a API do sistema de recomendação de pontos turísticos;\@
    implementação de um MVP do aplicativo móvel com funcionalidades
    de cadastro, questionário BFI, visualização de recomendações com indicadores de
    acessibilidade e sistema colaborativo de avaliação de pontos turísticos; validação
    através de testes de usabilidade com usuários reais. Espera-se contribuir com este projeto o fortalecimento
    da inclusão digital de pessoas idosas e com deficiência, demonstrando viabilidade de
    sistemas inteligentes que combinam algoritmos de recomendação com design centrado no usuário,
    alinhado com a Lei Brasileira de Inclusão (Lei 13.146/2015).

    \bigskip
    \noindent\textbf{Palavras-chave:} Aplicação móvel. Sistema de recomendação. Idoso. Turismo inclusivo. Acessibilidade.
\end{singlespace}

% \newpage
% \thispagestyle{empty}
% \section*{Abstract}
% \begin{singlespace} % espaçamento simples conforme NBR 6028
%     \justifying
%     This work presents the development of a mobile application for recommending tourist attractions with a focus on accessibility for elderly people and people with disabilities in São Luís - MA. The objective is to offer personalized recommendations adapted to the specific needs of users, promoting social and digital inclusion in the tourism context. The justification is based on the alarming fact that 53.5\% of tourists with disabilities fail to travel in Brazil due to lack of accessibility, and the absence of solutions that integrate personality with accessibility as a central element. The differential of the proposal lies in the simultaneous integration of the Big Five Inventory (BFI) psychological model with accessibility restrictions, generating recommendations that respect both individual preferences and guarantee physical feasibility and safety of the experience. The methodology follows Design Science Research with four phases: investigation of interface techniques for the elderly via systematic review and participatory research; Development of a REST API in Java Spring Boot for the application's business logic and integration with the API of the tourist attraction recommendation system; Implementation of an MVP of the mobile application with functionalities such as registration, BFI questionnaire, visualization of recommendations with accessibility indicators and a collaborative system for evaluating tourist attractions; validation through usability tests with real users. This project is expected to contribute to strengthening the digital inclusion of elderly and disabled people, demonstrating the viability of intelligent systems that combine recommendation algorithms with user-centered design, aligned with the Brazilian Inclusion Law (Law 13.146/2015).

%     \bigskip
%     \noindent\textbf{Keywords:} Mobile Application. Recommendation System. Elderly. Inclusive Tourism.
%     Accessibility.

% \end{singlespace}
