\documentclass[12pt,a4paper]{article}

% ---------------------------------------------------------------
% PACOTES BÁSICOS E FORMATAÇÃO ABNT
% ---------------------------------------------------------------
\usepackage[brazil]{babel}
\usepackage[utf8]{inputenc}
\usepackage[T1]{fontenc}
\usepackage[a4paper,top=3cm,bottom=2cm,left=3cm,right=2cm]{geometry}
\usepackage{setspace}
\usepackage{times}
\usepackage{indentfirst}
\usepackage{graphicx}
\usepackage{float}
\usepackage{hyperref}
\usepackage{enumitem}
\usepackage{titlesec}
\usepackage{ragged2e}
\usepackage{caption}
\usepackage[
  style=abnt,
  uniquename=init,
  giveninits,
  uniquelist=true,
  maxbibnames=4,
  repeatfields=true,
  justify
]{biblatex}
\addbibresource{bibliografia.bib}

% ---------------------------------------------------------------
% REGRAS DE FORMATAÇÃO
% ---------------------------------------------------------------
\setlength{\parindent}{1.25cm}
\setlength{\parskip}{0pt}
\onehalfspacing % espaçamento 1,5 conforme ABNT

% Estilo das seções ABNT (em maiúsculas e negrito)
\titleformat{\section}{\bfseries\normalsize\uppercase}{\thesection.}{0.5em}{}
\titleformat{\subsection}{\bfseries\normalsize}{\thesubsection.}{0.5em}{}
\titleformat{\subsubsection}{\normalsize}{\thesubsubsection.}{0.5em}{}

% Espaçamento entre título e texto
\titlespacing*{\section}{0pt}{1.5em}{1em}
\titlespacing*{\subsection}{0pt}{1em}{0.5em}

% Legendas em formato ABNT
\captionsetup{font=small,labelfont=bf,labelsep=endash}

% ---------------------------------------------------------------
% CAPA
% ---------------------------------------------------------------
\begin{document}
\thispagestyle{empty}
\begin{center}
    \begin{center}
        \includegraphics{logotipos/DcompIfma.png}
    \end{center}
    \singlespacing
    \textbf{INSTITUTO FEDERAL DE EDUCAÇÃO, CIÊNCIA E TECNOLOGIA DO MARANHÃO}\\
    \textbf{CAMPUS SÃO LUÍS - MONTE CASTELO}\\
    \textbf{DEPARTAMENTO DE COMPUTAÇÃO}\\
    \textbf{CURSO DE BACHARELADO EM SISTEMAS DE INFORMAÇÃO}\\[4cm]

    \textbf{PAULO HENRIQUE SILVA PINHEIRO}\\[4cm]

    \textbf{\MakeUppercase{Desenvolvimento de um aplicativo móvel integrado a um sistema de recomendação para turismo acessível com foco na pessoa idosa: Estudo de caso Persona Tour}}\\[4cm]

    \vfill
    \textbf{SÃO LUÍS – MA}\\
    \textbf{2025}
\end{center}

% ---------------------------------------------------------------
% FOLHA DE ROSTO
% ---------------------------------------------------------------
\newpage
\thispagestyle{empty}
\begin{center}
    \textbf{PAULO HENRIQUE SILVA PINHEIRO}\\[3cm]

    \textbf{\MakeUppercase{Desenvolvimento de um aplicativo mobile integrando um sistema de recomendação para turismo inclusivo com foco na pessoa idosa: Estudo de caso Persona Tour}}\\[2cm]
\end{center}

\begin{flushright}
    \begin{minipage}{0.6\textwidth}
        \justifying
        \noindent
        Projeto de Trabalho de Conclusão de Curso apresentado à disciplina \textbf{Monografia I}, ministrada pelo Prof. Dr. Hélder Pereira Borges, como requisito básico para o Trabalho de Conclusão de Curso.\\[0.25cm]

        \noindent
        Orientadora: Profa. Dra. Eveline de Jesus Viana Sá
    \end{minipage}
\end{flushright}

\vfill
\begin{center}
    \textbf{SÃO LUÍS – MA}\\
    \textbf{2025}
\end{center}

% ---------------------------------------------------------------
% RESUMO
% ---------------------------------------------------------------
\newpage
\thispagestyle{empty}
\section*{RESUMO}
\begin{singlespace} % espaçamento simples conforme NBR 6028
    \justifying
    Este trabalho apresenta o desenvolvimento de um aplicativo móvel voltado ao turismo inclusivo, O objetivo é oferecer recomendações personalizadas de pontos turísticos (PTs) adaptadas às necessidades de pessoas idosas e com deficiência, para a cidade de São Luís - MA, promovendo acessibilidade e inclusão digital. Fundamentado no modelo psicológico Big Five Inventory (BFI), o sistema considera traços de personalidade, preferências e motivações de viagem para gerar recomendações mais precisas e relevantes a partir de um questionário. Diferentemente das abordagens convencionais, que se limitam somente a fatores demográficos ou históricos de navegação, esta proposta busca compreender o perfil psicológico do usuário e suas limitações específicas, aprimorando a experiência turística de forma personalizada. O aplicativo busca integrar o sistema de recomendação desenvolvido em projeto de iniciação científica, inspirado no sistema de recomendação para grupos GrouPlanner, por sua vez implementado em Python, possibilitando a recomendação de pontos turísticos adequadas ao público idoso. Espera-se que a aplicação contribua para o fortalecimento da inclusão social e digital de pessoas idosas e com deficiência, ampliando sua participação em atividades turísticas e culturais e promovendo o acesso equitativo à esses locais.

    \bigskip
    \noindent\textbf{Palavras-chave:} Aplicação mobile. Sistema de recomendação. Idoso. Turismo inclusivo. Acessibilidade.
\end{singlespace}

\newpage
\thispagestyle{empty}
\section*{Abstract}
\begin{singlespace} % espaçamento simples conforme NBR 6028
    \justifying
    This work presents the development of a mobile application focused on inclusive tourism. The objective is to offer personalized recommendations of tourist attractions (TAs) adapted to the needs of elderly people and people with disabilities, promoting accessibility and digital inclusion. Based on the Big Five Inventory (BFI) psychological model, the system considers personality traits, preferences, and travel motivations to generate more accurate and relevant recommendations. Unlike conventional approaches, which are limited to demographic factors or browsing history, this proposal seeks to understand the user's psychological profile and their specific limitations, improving the tourist experience in a personalized way. The application seeks to integrate the recommendation system developed in an undergraduate research project, inspired by the GrouPlanner group recommendation system, which in turn is implemented in Python, enabling the recommendation of tourist attractions suitable for the elderly public. It is expected that the application will contribute to strengthening the social and digital inclusion of elderly people and people with disabilities, expanding their participation in tourist and cultural activities and promoting equitable access to these places.

    \bigskip
    \noindent\textbf{Keywords:} Mobile Application. Recommendation System. Elderly. Inclusive Tourism. Accessibility.

\end{singlespace}

% ---------------------------------------------------------------
% SUMÁRIO
% ---------------------------------------------------------------
\newpage
\thispagestyle{empty}
\tableofcontents
\newpage

% ---------------------------------------------------------------
% 1 INTRODUÇÃO
% ---------------------------------------------------------------
\newpage
\section{Introdução}
\justifying

(TEXTO DO PROBLEMA)

Em 2023 teve início o projeto de pós-doutorado intitulado "Uma análise do uso
de IA na geração de rotas turísticas personalizadas para o público idoso: um
estudo de caso no Projeto The Route", desenvolvido no Instituto Superior de
Engenharia do Porto (ISEP/IPP), Portugal, de julho de 2023 a junho de 2024. O
Projeto The Route \cite{faria2019theroute} evoluiu para o GrouPlanner
\cite{alves2022grouplanner}, coordenado pelo GECAD/ISEP-IPP, incorporando
recomendações de roteiros para grupos baseadas nos traços de personalidade do
modelo Big Five \cite{Costa1992} e técnicas de negociação para lidar com
heterogeneidade grupal na região do Porto e norte de Portugal
\cite{alves2023group}.

Inspirado nesse contexto e na evolução das pesquisas conduzidas pelo ISEP-IPP,
Estudou-se a viabilidade de aplicar tais conceitos ao turismo acessível no
Brasil, com ênfase na população idosa. Assim, um projeto de iniciação
científica foi desenvolvido no Instituto Federal do Maranhão (IFMA) Campus São
Luís - Monte CASTELO entre agosto de 2024 e agosto de 2025, intitulado "Um
estudo sobre os algoritmos inteligentes utilizados no Sistema de Recomendação
para Grupos aplicado ao Turismo: um estudo de caso do Projeto GrouPlanner" O
projeto resultou na implementação de um sistema de recomendação o qual foi
nomeado Pesona Tour, baseado no algoritmo \textit{d-means} \cite{Alves2024},
adaptado para considerar aspectos de acessibilidade em pontos turísticos na
cidade de São Luís - MA. O sistema utiliza o modelo BFI para agrupar usuários
com perfis psicológicos semelhantes, recomendando pontos turísticos que atendam
às suas preferências e necessidades específicas de acessibilidade. Com base nos
resultados positivos obtidos nesse projeto inicial, o presente trabalho propõe
a continuidade e adaptação desses avanços tecnológicos por meio do
desenvolvimento do aplicativo para o serviço Persona Tour. O objetivo central
consiste em desenvolver um aplicação móvel voltada ao turismo acessível que faz
o uso do sistema de recomendação turística para obtenção dos pontos turísticos
, com especial atenção às necessidades do público idoso.O aplicativo visa
oferecer uma experiência turística personalizada, acessível e sensível às
limitações e preferências do usuário, utilizando o modelo psicológico BFI (Big
Five Inventory – BFI) e uma rede colaborativa dentro do App reforçando os
recursos disponíveis em pontos turisitcos.

Deste modo este trabalho configura-se como uma evolução metodológica e
tecnológica no campo do turismo acessível, combinando algoritmos de
recomendação não supervisionada com princípios de design centrado no usuário. O
projeto dialoga com políticas públicas de inclusão social, respondendo a uma
demanda crescente por soluções tecnológicas humanizadas, que considerem as
diversidades físicas, cognitivas e emocionais dos indivíduos.

% ---------------------------------------------------------------
% 1.1 PROBLEMA
% ---------------------------------------------------------------
\newpage
\subsection{Problema}
\justifying
A acessibilidade é um fator decisivo para a efetiva inclusão no
turismo \cite{unwto2016manual}. Para turistas com deficiência
ou mobilidade reduzida, não basta apenas saber “o que visitar”,
mas se os locais dispõem de recursos como rampas, banheiros
adaptados, piso tátil, sinalização adequada ou rotas sem degraus,
informações que raramente aparecem nos catálogos disponíveis de
forma confiável e comparável. Essa lacuna gera barreiras à autonomia
dos viajantes e reduz o interesse em locais que não evidenciam informações
sobre acessibilidade. Ainda assim, o tema da acessibilidadepermanece como um dos
aspectos menos contemplados pela maioria dos sistemas de
recomendação turísticos atuais \cite{santos2019poiaccessibility}.
Nesse sentido, a personalização aliada à acessibilidade
torna-se essencial para melhorar a experiência turística e ampliar a
competitividade econômica do setor \cite{Silva2023}

Portanto, esta pesquisa se propõe a responder à seguinte indagação central:
Como elaborar uma aplicação capaz de satisfazer as necessidades de pessoas com
deficiência e idosos permitindo o acesso a informação de pontos turisitcos
baseados em suas preferências e personalidade e promovendo a devida inclusão
social, turística e digital?.

% ---------------------------------------------------------------
% 1.2 JUSTIFICATIVA
% ---------------------------------------------------------------
\newpage
\subsection{Justificativa}
\justifying
Avançando nesse caminho, a presente proposta insere a acessibilidade como eixo
central da personalização,
transformando informações sobre barreiras e recursos em sinais processados pelo
algoritmo para a recomendação de pontos turísticos coesos com o usuário. Assim,
não apenas pontos turísticos são sugeridos, mas condições de acessibilidade do
local e preferências individuais do usuário e do grupo.
Com isso, o projeto justifica-se pela necessidade de superar as limitações dos
recomendadores tradicionais, que priorizam popularidade e cliques,
mas desconsideram variáveis críticas de inclusão \cite{Azambuja2021}.
Social e legalmente, a proposta materializa princípios da Lei Brasileira de
Inclusão ao trazer a acessibilidade para o centro do processo decisório
, academicamente, contribui para a inovação em sistemas de necessidade
social \cite{brasil2014turismoacessivel} \cite{brasil2015lei13146}.

Diante desse contexto, a proposta do Persona Tour justifica-se pela necessidade
de promover a inclusão digital e social da pessoa idosa e da pessoa com
deficiência, no turismo, utilizando a tecnologia como mediadora de experiências
significativas, seguras e adaptadas às suas limitações e interesses. O
aplicativo busca não apenas recomendar destinos e pontos de interesse, mas
também compreender as preferências e necessidades de cada usuário, propondo
lugares compatíveis com suas motivações, nível de atividade e preferências
categóricas. Além disso, a interface do aplicativo foca em ser acessível, com
elementos visuais, tipográficos e interativos adaptados à ergonomia e à
usabilidade de pessoas idosas, contribuindo para a redução das barreiras de
acesso à informação turística.

% ---------------------------------------------------------------
% 1.3 OBJETIVOS
% ---------------------------------------------------------------
\newpage
\subsection{Objetivos}
\subsubsection{Objetivo Geral}
Desenvolver um aplicativo móvel de recomendação de pontos turísticos,
delimitado na recomendação de locais da cidade de São Luís, com foco na
acessibilidade pessoas com deficiência e idosos.

\subsubsection{Objetivos Específicos}
\begin{itemize}
    \item[a.] Investigar técnicas de interface intuitiva para os usuários idosos;
    \item[b.] Desenvolver uma API para integrar o sistema de recomendação ao aplicativo
          móvel.
    \item[c.] Implementar um mínimo produto viável do aplicativo móvel, oferecendo
          interface intuitiva que permita a visualização interativa de pontos turísticos.
    \item[d.] Avaliar a eficácia do aplicativo por meio de testes com diferentes perfis
          de usuários, verificando a precisão das recomendações, a adequação às condições
          de acessibilidade e a percepção de usabilidade.
\end{itemize}

% ---------------------------------------------------------------
% 2 REFERENCIAL TEÓRICO
% ---------------------------------------------------------------
\newpage
\section{Referencial Teórico}
\justifying
\subsection{Turismo Acessível e Acessibilidade}

O turismo representa uma das atividades econômicas e sociais mais importantes
para países em desenvolvimento, funcionando como gerador de renda, emprego e
intercâmbio cultural. Contudo, a participação igualitária nessa prática depende
fundamentalmente da existência de infraestrutura e serviços adequados que
considerem as necessidades específicas de pessoas com deficiência e mobilidade
reduzida, incluindo idosos. A acessibilidade turística transcende a
disponibilidade de rampas e banheiros adaptados; ela envolve um conceito mais
amplo que contempla elementos informativos, comunicacionais, tecnológicos e
atitudinais.

Segundo dados recentes do Ministério do Turismo, mais de 53,5\% dos turistas
com deficiência deixaram de viajar para algum destino no Brasil por falta de
acessibilidade, revelando uma exclusão sistemática desse
público~\cite{MTur2024}. O Brasil, embora reconhecido internacionalmente por
suas belezas naturais, apresenta graves deficiências na operacionalização de
políticas de turismo inclusivo, conforme demonstrado em análises de políticas
públicas dos últimos 20 anos~\cite{Vilela2022}.

Um caso emblemático dessa falha institucional é o ``Guia de Rodas'', projeto
desenvolvido pelo próprio governo brasileiro para mapear acessibilidade em
pontos turísticos. O aplicativo, embora bem-intencionado, apresenta limitações
críticas: dados desatualizados, informações inconsistentes sobre acessibilidade
real dos espaços, ausência de validação contínua pelas comunidades de pessoas
com deficiência, e interface pouco acessível para o público idoso. Muitos
pontos turísticos cadastrados como acessíveis carecem de infraestrutura
essencial (rampas inadequadas, banheiros sem adaptações, falta de sinalização
tátil), demonstrando que a presença de informação não garante qualidade de
implementação.

A Lei Brasileira de Inclusão da Pessoa com Deficiência (Lei 13.146/2015) e o
Manual de Turismo Acessível da UNWTO estabelecem que acessibilidade é direito
fundamental, não benefício~\cite{Brasil2015, UNWTO2016}. Entretanto, a
implementação permanece lenta e desigual. O conceito de turismo acessível deve
ser compreendido em múltiplas dimensões:

\begin{itemize}
    \item \textbf{Acessibilidade Arquitetônica}: Remoção de barreiras físicas (rampas adequadas com inclinação máxima de 8,33\%, corrimões, elevadores, pisos táteis, banheiros com espaço de manobra mínimo de 1,5m de diâmetro).

    \item \textbf{Acessibilidade Comunicacional}: Disponibilidade de informações em formatos alternativos (Braille, áudio, linguagem simples, descrição de imagens), sinalização clara e legível, descrição de ambientes.

    \item \textbf{Acessibilidade Atitudinal}: Treinamento de profissionais de turismo para atender com qualidade e respeito, compreensão da neurodiversidade e das diferentes formas de deficiência, valorização da autonomia do visitante.

    \item \textbf{Acessibilidade Tecnológica}: Disponibilidade de recursos assistivos (leitores de tela, amplificadores, sistemas de amplificação de som), integração com tecnologias móveis, interfaces acessíveis em plataformas digitais.
\end{itemize}

\subsection{Sistemas de Recomendação no Turismo}

Os Sistemas de Recomendação (SR) têm revolucionado a forma como os indivíduos
descobrem produtos, serviços e informações em ambientes digitais. No contexto
turístico, eles auxiliam viajantes a identificar pontos de interesse relevantes
em meio à sobrecarga informacional, reduzindo significativamente o tempo de
pesquisa e aumentando a qualidade da experiência~\cite{Resnick1997,
    Santos2023}. Um sistema de recomendação funciona como intermediário entre a
diversidade de ofertas turísticas e as preferências específicas de cada
usuário, processando dados comportamentais, demográficos e psicológicos para
gerar sugestões personalizadas.

As abordagens tradicionais em Sistemas de Recomendação Turística baseiam-se
predominantemente em duas técnicas: \textbf{filtragem colaborativa}, que
identifica usuários com preferências similares e recomenda itens apreciados por
esses ``vizinhos''; e \textbf{filtragem baseada em conteúdo}, que analisa
características dos pontos turísticos (categoria, localização, características)
e as compara com o perfil do usuário~\cite{Resnick1997}. Contudo, essas
abordagens possuem limitações reconhecidas: a filtragem colaborativa sofre com
o problema de ``partida a frio'' (poucos dados históricos de usuários novos),
enquanto a filtragem por conteúdo tende a criar ``bolhas de informação'' que
reduzem a serendipidade.

O \textbf{GrouPlanner}~\cite{Alves2022, Alves2023}, desenvolvido pelo
GECAD/ISEP-IPP (Grupo de Pesquisa em Engenharia Computacional de Sistemas
Adaptativos Complexos do Instituto Superior de Engenharia do Porto), representa
um avanço significativo ao incorporar \textbf{modelagem de personalidade} como
base para recomendação. O GrouPlanner utiliza o modelo psicológico \textbf{Big
    Five Inventory (BFI)}~\cite{Costa1992} para capturar os cinco principais traços
de personalidade humana:

\begin{enumerate}
    \item \textbf{Abertura para Experiências}: Propensão para novelidade, criatividade e flexibilidade. Usuários com alta abertura buscam atrações inovadoras, culturais e desafiadoras.

    \item \textbf{Conscienciosidade}: Organização, planejamento e atenção a detalhes. Influencia a preferência por roteiros estruturados e segurança.

    \item \textbf{Extroversão}: Sociabilidade e busca por estimulação. Determina preferência por atividades em grupo, vida noturna ou contemplação solitária.

    \item \textbf{Agradabilidade}: Orientação para cooperação e harmonia. Afeta disposição para turismo comunitário e experiências inclusivas.

    \item \textbf{Neuroticismo}: Tendência para emoções negativas. Relaciona-se com sensibilidade a stress ambiental, necessidade de conforto e segurança.
\end{enumerate}

O BFI é implementado através de um questionário validado cientificamente
(~\cite{Costa1992, Andrade2023}) que permite classificar usuários em um espaço
multidimensional de personalidade. A principal inovação do GrouPlanner reside
na aplicação do \textbf{algoritmo d-means}~\cite{Alves2024} para agrupar
usuários com perfis psicológicos similares e gerar recomendações coletivas em
contextos de grupo, negociando preferências heterogêneas de forma harmoniosa.

Para o Persona Tour, a arquitetura de recomendação foi adaptada do GrouPlanner,
incorporando uma dimensão crítica ausente em sistemas tradicionais:
\textbf{modelagem de restrições de acessibilidade}. O sistema processa não
apenas traços de personalidade, mas também informações sobre:

\begin{itemize}
    \item Tipo e grau de deficiência/mobilidade (cadeira de rodas, deficiência visual,
          auditiva, cognitiva)
    \item Necessidades específicas (presença de elevador, piso tátil, intérprete de
          Libras, descrição de ambientes)
    \item Preferências de tipo de turismo (cultural, natural, gastronômico, aventura)
          ajustadas às capacidades do usuário
\end{itemize}

A integração entre BFI e restrições de acessibilidade cria um modelo de
recomendação verdadeiramente inclusivo, que não apenas respeita preferências de
personalidade mas também garante a viabilidade física e segurança da
experiência turística. Esse diferencial posiciona o Persona Tour como inovação
no campo do turismo inteligente (Tourism 4.0), alinhado com políticas públicas
de inclusão social~\cite{Silva2023}.

\subsection{Aplicações Mobile e Experiência do Usuário (UX)}

A proliferação de dispositivos móveis transformou a forma como as pessoas
interagem com informações, serviços e experiências. Aplicativos móveis
representam 90\% do tempo de engajamento digital, tornando-os o canal
preferencial para acesso a serviços turísticos. Contudo, a qualidade e
acessibilidade dessas aplicações variam enormemente, com muitas falhando em
atender necessidades de usuários com deficiência ou idosos~\cite{Bevan2021,
Bruffaerts2021}.

\subsubsection{Acessibilidade em Aplicações Móveis: WCAG 2.1}

As \textbf{Web Content Accessibility Guidelines 2.1 (WCAG 2.1)}~\cite{W3C2021},
desenvolvidas pelo World Wide Web Consortium, estabelecem padrões
internacionais reconhecidos para acessibilidade digital. Embora originalmente
focadas em conteúdo web, suas diretrizes são amplamente aplicáveis a aplicações
móveis, especialmente em nível AA (que garante acessibilidade para 80\% das
necessidades de pessoas com deficiência)~\cite{W3C2021, MDPI2021}.

Os princípios WCAG 2.1 fundamentais para aplicações móveis são:

\begin{enumerate}
    \item \textbf{Perceptível}: A informação deve ser apresentada de forma compreensível. Especificamente para aplicações:
          \begin{itemize}
              \item \textbf{Contraste de cores}: Razão mínima de 4,5:1 entre texto e fundo (ou 3:1 para texto grande $\geq$ 18pt). Aplicações frequentemente violam esse requisito, causando ilegibilidade para pessoas com baixa visão~\cite{Acessibilidade2025}.
              \item \textbf{Tamanho de texto}: Deve ser ajustável e a interface deve permanecer utilizável em escalas até 200\%.
              \item \textbf{Descrição de elementos visuais}: Botões, ícones e imagens devem ter rótulos textuais ou descrições acessíveis para leitores de tela.
          \end{itemize}

    \item \textbf{Operável}: Todos os componentes devem ser operáveis sem exclusão de usuários com diferentes capacidades motoras ou cognitivas.
          \begin{itemize}
              \item \textbf{Tamanho de alvo tátil}: Botões e elementos interativos devem ter área mínima de 48x48 pixels lógicos para facilitar ativação por usuários com limitações motoras finas~\cite{Flutter2025, Dev2023}.
              \item \textbf{Navegação por teclado}: Toda funcionalidade deve ser acessível via teclado, suportando navegação com abas e foco claramente visível.
              \item \textbf{Suporte a leitores de tela}: TalkBack (Android) e VoiceOver (iOS) devem descrever a interface de forma coerente e permitir navegação completa~\cite{Flutter2025}.
              \item \textbf{Evitar mudanças de contexto automáticas}: Alterações não deve ocorrer sem confirmação do usuário, especialmente durante entrada de dados.
          \end{itemize}

    \item \textbf{Compreensível}: Interface, comportamento e mensagens devem ser claros e antecipáveis.
          \begin{itemize}
              \item \textbf{Linguagem simples}: Texto legível com vocabulário apropriado, especialmente importante para usuários idosos e com deficiência cognitiva~\cite{Recomendacoes2015}.
              \item \textbf{Feedback claro}: Ações do usuário devem gerar feedback imediato (visual, auditivo ou tátil) confirmando sucesso ou indicando erro com sugestão de correção.
              \item \textbf{Consistência visual}: Padrões, cores e posicionamento devem ser consistentes ao longo da aplicação.
          \end{itemize}

    \item \textbf{Robusto}: Deve funcionar com diversas tecnologias assistivas e em diferentes contextos.
          \begin{itemize}
              \item \textbf{Compatibilidade com tecnologias assistivas}: Deve funcionar adequadamente com leitores de tela, amplificadores de som, interfaces adaptadas.
              \item \textbf{Testes em múltiplos dispositivos}: Comportamento consistente em smartphones, tablets, versões antigas e novas do SO.
          \end{itemize}
\end{enumerate}

\subsubsection{Recomendações Específicas para Idosos}

Estudos sobre interação de idosos com aplicações
móveis~~\cite{Recomendacoes2015, LUSUI2021, Vilela2022} identificam desafios
cognitivos, visuais e motores específicos que exigem diretrizes adaptadas:

\begin{itemize}
    \item \textbf{Degradação visual natural}: Redução de acuidade visual, dificuldade com contraste e aumento de sensibilidade a cintilação. Recomenda-se font-size mínimo de 16-18pt, espaçamento entre elementos, aumento de espaçamento de linha.

    \item \textbf{Redução de destreza motora}: Tremor, artrite e redução de precisão manual. Botões devem ser significativamente maiores que 48x48 pixels (60-72 pixels recomendado), com espaçamento amplo entre elementos~\cite{Recomendacoes2015}.

    \item \textbf{Mudança cognitiva}: Aumento de tempo de processamento, dificuldade com abstrações visuais, tendência a familiaridade com padrões conhecidos. Navegação deve ser simplificada (máximo 3 níveis hierárquicos), com nomes de botões claros e descritivos, evitando ícones sem rótulo.

    \item \textbf{Proficiência tecnológica variável}: Muitos idosos possuem experiência limitada com aplicativos móveis. Deve-se priorizar familiaridade (padrões conhecidos como enviar/receber para comunicação), tutoriais integrados e suporte à ajuda contextual.
\end{itemize}

\subsubsection{Experiência do Usuário (UX) Centrada no Usuário Específico}

A Experiência do Usuário vai além da mera funcionalidade; refere-se ao conjunto
de emoções, atitudes e comportamentos associados ao uso de um
sistema~\cite{Nielsen2012}. Para públicos com necessidades especiais, UX deve
ser construída através de pesquisa participativa, onde o usuário é co-designer
do sistema.

O \textbf{Design Thinking} propõe que aplicações sejam desenvolvidas em
iterações: (1) Descoberta --- entender profundamente as necessidades reais; (2)
Definição --- reformular o problema com base em insights; (3) Ideação --- gerar
múltiplas soluções; (4) Prototipação --- construir versões tangíveis; (5) Teste
--- validar com usuários reais. Para aplicações turísticas, isso significa
envolver desde o início pessoas idosas e com deficiência em sessões de
prototipagem e teste de usabilidade, capturando não apenas erros funcionais,
mas também a satisfação emocional e o senso de autonomia gerado pela
experiência.

\subsection{Tecnologias e Arquitetura do Sistema}

O Persona Tour é uma arquitetura de três camadas que separa responsabilidades e
permite escalabilidade:

\subsubsection{Camada 1: API de Integração de Dados Geográficos --- Google Maps API}

A primeira camada responsabiliza-se por agregar dados de pontos turísticos. A
\textbf{Google Maps API} fornece informações geográficas confiáveis:
localização (latitude/longitude), nome, categoria (restaurante, atração
cultural, museu, etc.), avaliações, horários, contato. Esses dados brutos são
enriquecidos através de:

\begin{itemize}
    \item \textbf{Anotação manual}: Time de pesquisadores verificam in loco a acessibilidade de cada ponto, registrando informações específicas (presença de rampas, dimensões de entradas, tipo de piso, elevadores, banheiros adaptados, sinalização tátil/Braille, serviços de acessibilidade específicos como atendimento personalizado).

    \item \textbf{Crowdsourcing validado}: Usuários do Persona Tour (especialmente pessoas com deficiência) podem relatar discrepâncias entre informações cadastradas e realidade, com mecanismo de validação por pares e moderação.

    \item \textbf{Integração com bases de dados públicas}: Dados de entidades turísticas oficiais, secretarias municipais de turismo e cultura.
\end{itemize}

\subsubsection{Camada 2: API de Negócios e Recomendação}

A segunda camada implementa a lógica de negócio central: processamento de
perfil do usuário, geração de recomendações e gerenciamento de dados de
acessibilidade. Estrutura-se em:

\begin{enumerate}
    \item \textbf{Módulo de Autenticação e Perfil}: Gerencia cadastro de usuários, armazenamento seguro de credenciais (com criptografia), e persistência de dados de perfil (traços BFI, restrições de acessibilidade, histórico de visitas).

    \item \textbf{Módulo de Questionário BFI}: Implementa o Big Five Inventory validado em português brasileiro. O questionário composto por 44 ou 15 itens (versão reduzida) coleta respostas em escala Likert de 5 pontos, calculando scores para cada dimensão de personalidade.

    \item \textbf{Módulo de Recomendação}:
          \begin{itemize}
              \item Recebe perfil BFI do usuário (5 dimensões), restrições de acessibilidade, e
                    preferências de categoria turística.
              \item Aplica algoritmo d-means inspirado do GrouPlanner para agrupar pontos turísticos
                    que combinam: (a) características que atraem usuários com aquele perfil
                    psicológico; (b) conformidade com requisitos de acessibilidade específicos do
                    usuário.
              \item Retorna lista ordenada de pontos turísticos recomendados com score de relevância.
          \end{itemize}

    \item \textbf{Módulo de Dados de Acessibilidade}: Armazena e gerencia metadados detalhados sobre cada ponto turístico, permitindo buscas facetadas por tipo de acessibilidade necessária.
\end{enumerate}

A API de Negócios é implementada em Java Spring Boot, com requisitos de segurança rígidos
(autenticação JWT, validação de entrada, CORS), e padrão RESTful.

\subsubsection{Camada 3: Aplicação Móvel --- Flutter}

A terceira camada é a interface de usuário, implementada em \textbf{Flutter},
framework open-source de desenvolvimento cross-platform da Google que permite
criar aplicações nativas em iOS e Android com código base único
\cite{Flutter2025}.

\textbf{Razões para escolha do Flutter}:

\begin{itemize}
    \item \textbf{Suporte nativo a acessibilidade}: Framework Flutter possui integração built-in com tecnologias assistivas do SO: TalkBack (Android) e VoiceOver (iOS). Widgets fornecem propriedades como `semanticLabel`, `onTap`, `onLongPress` que permitem navegação por leitor de tela \cite{Flutter2025}.

    \item \textbf{Widgets acessíveis por padrão}: Componentes como `Button`, `TextField`, `AppBar` já implementam acessibilidade mínima, facilitando a implementação por parte do desenvolvedor \cite{Flutter2025}.

    \item \textbf{WCAG 2.1 Compliance}: Pacotes como \texttt{flutter\_accessibility\_scanner} permitem verificação automática de conformidade com as diretrizes WCAG~2.1, detectando problemas como contraste inadequado, alvo tátil pequeno, rótulos ausentes e falhas na navegação por teclado.

    \item \textbf{Performance e responsividade}: Flutter compila para código nativo, oferecendo performance superior a tecnologias interpretadas, crucial para experiências fluidas em dispositivos antigos/com recursos limitados, comuns entre usuários idosos.

    \item \textbf{Hot Reload}: Permite iteração rápida durante desenvolvimento e teste de acessibilidade.
\end{itemize}

\textbf{Componentes principais da aplicação Flutter}:

\begin{enumerate}
    \item \textbf{Tela de Autenticação}: Login/registro com campos amplos (48+ pixels), alta relação de contraste, feedback claro de erros, suporte a navegação por teclado completa.

    \item \textbf{Tela de Questionário BFI}: Interface em múltiplas páginas (não carregar todas as 44 questões de uma vez para evitar sobrecarga cognitiva), com indicador de progresso claro, opção de pausar e retomar, escala Likert com rótulos textuais claros (``Discordo Fortemente'' a ``Concordo Fortemente'').

    \item \textbf{Tela de Recomendações}: Exibe lista de pontos turísticos recomendados em card layout acessível, cada card mostrando: foto (com descrição alternativa para leitores de tela), nome, categoria, distância, e indicadores de acessibilidade (ícones com rótulos: ``Rampa disponível'', ``Banheiro adaptado'', ``Estacionamento próximo'').

    \item \textbf{Tela de Detalhes do Ponto Turístico}: Informação expandida: endereço completo com mapa interativo, descrição detalhada (com audiodescrição de fotos disponível), informações de acessibilidade, contato, horários, como chegar (com sugestões de transporte acessível), avaliações de outros usuários, galeria de fotos.

    \item \textbf{Tela de Avaliação de Acessibilidade do Ponto Turístico}: Informação expandida: endereço completo com mapa interativo, descrição detalhada (com audiodescrição de fotos disponível), informações de acessibilidade, contato, horários, como chegar (com sugestões de transporte acessível), avaliações de outros usuários, galeria de fotos.

    \item \textbf{Perfil do Usuário}: Permite gerenciar informações pessoais, restrições de acessibilidade, histórico de visitas, favoritos, e configurações de acessibilidade (tamanho de fonte, modo de alto contraste, velocidade de animação).
\end{enumerate}

\textbf{Implementação de Acessibilidade em Flutter}:

Seguindo as diretrizes WCAG 2.1 e recomendações específicas para idosos:

\begin{itemize}
    \item \textbf{Tipografia}: Font-size mínimo 16pt para corpo de texto, 18-20pt para cabeçalhos, fonte sans-serif (Roboto ou Open Sans) de alta legibilidade. Suporte a aumento de escala até 200\%.

    \item \textbf{Cores e Contraste}: Paleta de cores com contraste mínimo 4,5:1. Modo de alto contraste disponível. Não usar cor como único indicador de informação (usar também ícones, texto, padrões).

    \item \textbf{Navegação}: Estrutura hierárquica clara (máximo 3 níveis), botões ``Voltar'' conspícuos, breadcrumbs em páginas profundas, suporte a navegação por teclado com ordem de foco lógica.

    \item \textbf{Leitores de Tela}: Todos os elementos interativos possuem `semanticLabel` descritivo. Imagens têm `semanticLabel` explicativo. Ordem semântica (reading order) otimizada para fluxo natural. Testado extensivamente com TalkBack e VoiceOver.

    \item \textbf{Feedback}: Interações geram feedback multissensorial: visual (mudança de cor, ícone), auditivo (som, vibração), textual (toast/snackbar explicando resultado).

    \item \textbf{Erros e Validação}: Campos obrigatórios claramente marcados. Erros de validação aparecem imediatamente com sugestão de correção. Importante ações são reversíveis (ex: ``Desfazer último filtro'').
\end{itemize}

\textbf{Integração das Camadas}:

A aplicação Flutter comunica-se com a API de Negócios através de requisições
HTTPS/REST, transmitindo: - Credenciais autenticadas (JWT token) - Dados de
perfil (scores BFI, restrições de acessibilidade) - Preferências de filtro -
Localização atual (com consentimento do usuário)

A API retorna: - Lista de recomendações com metadata - Detalhes expandidos de
cada ponto - Informações de acessibilidade validadas - Rotas otimizadas

Ambas as camadas (API e Flutter) interagem com Google Maps API para dados
geográficos.

\textbf{Segurança e Privacidade}:

- Dados sensíveis (localização, informações de deficiência) são criptografados em trânsito (TLS 1.3) e em repouso (AES-256).
- Usuários controlam quais dados compartilham e com quem.
- Conformidade com LGPD (Lei Geral de Proteção de Dados) garantida através de consentimento explícito, direito de acesso/exclusão de dados.
- Auditorias regulares de segurança e testes de penetração.

Essa arquitetura em três camadas garante separação de responsabilidades,
escalabilidade, manutenibilidade e, crucialmente, permite que melhorias em uma
camada (ex: novos algoritmos de recomendação, novos dados de acessibilidade)
beneficiem imediatamente toda a plataforma.

% \begin{figure}[H]
%     \centering
%     %    \includegraphics[width=0.8\textwidth]{estrutura_sistema.png}
%     \caption{Relação entre módulos do sistema proposto e componentes teóricos.}
% \end{figure}

% ---------------------------------------------------------------
% 3 METODOLOGIA
% ---------------------------------------------------------------
\newpage
\section{Metodologia}
\justifying
\subsection{DSR - Design Science Research}
Este trabalho caracteriza-se como uma pesquisa aplicada de natureza
exploratória e experimental, fundamentada nos princípios da Design Science
Research (DSR), metodologia reconhecida por sua capacidade de produzir
artefatos tecnológicos que resolvem problemas concretos enquanto geram
conhecimento científico. A pesquisa foi estruturada em quatro fases
interdependentes e iterativas, alinhadas aos objetivos específicos propostos,
integrando técnicas qualitativas e quantitativas para aplicabilidade prática.

\subsubsection{Fase 1: Investigação de Técnicas de Interface Intuitiva para Usuários Idosos}
A primeira fase concentra-se na compreensão profunda das necessidades e
limitações do público idoso no contexto de interfaces digitais móveis.
Iniciou-se com uma revisão sistemática da literatura abrangendo diretrizes de
acessibilidade para idosos, recomendações de usabilidade específicas para
dispositivos móveis e estudos sobre interação humano-computador com foco nesse
público. Foram consultadas as diretrizes WCAG 2.2 (Web Content Accessibility
Guidelines), adaptadas para o contexto mobile, bem como estudos que investigam
características visuais, cognitivas e motoras associadas ao envelhecimento.
Paralelamente à revisão bibliográfica, serão conduzidas entrevistas
semiestruturadas com potenciais usuários com deficiência e idosos, além de
pessoas que trabalham na área do turismo que atuam com esse público. As
entrevistas seguirão a metodologia de pesquisa participativa, permitindo
identificar preferências de design, barreiras de uso e expectativas
relacionadas a aplicativos turísticos. Os dados coletados deverão ser
organizados através de análise temática, gerando requisitos preliminares sobre
tamanho de fonte, contraste de cores, simplicidade de navegação, feedback
audiovisual e redução de passos para completar tarefas.

\subsubsection{Fase 2: Desenvolvimento da API para Integração com o Sistema de Recomendação}
A segunda fase envolverá o projeto e implementação de uma Application
Programming Interface (API) REST responsável pela comunicação entre o
aplicativo móvel e o sistema de recomendação Persona Tour desenvolvido em
Python. A arquitetura da API deverá ser projetada seguindo os princípios
RESTful, garantindo escalabilidade, manutenibilidade e segurança. O
desenvolvimento será realizado utilizando boas práticas de engenharia de
software, incluindo: definição clara de endpoints para operações de criação,
leitura, atualização e exclusão de dados (CRUD); implementação de autenticação
e autorização baseada em tokens JWT (JSON Web Tokens); versionamento da API
para garantir retrocompatibilidade; e testes automatizados de integração. A API
foi estruturada para receber como entrada os dados do questionário BFI (Big
Five Inventory) preenchido pelo usuário, e enviar essas informações a API do
Sistema de Recomendação, através do algoritmo d-means adaptado do sistema
Persona Tour, recomendações personalizadas de pontos turísticos com informações
detalhadas sobre os locais são retornadas para a API em Java Spring Boot. A
validação da API incluirá testes de carga para avaliar desempenho sob
diferentes volumes de requisições simultâneas, testes de segurança para
identificar vulnerabilidades, e testes funcionais para garantir a correta
integração com o sistema de recomendação.

\subsubsection{Fase 3: Implementação do Mínimo Produto Viável (MVP) do Aplicativo Móvel}
A terceira fase consistirá no desenvolvimento do MVP do aplicativo, seguindo a
metodologia ágil de desenvolvimento. A escolha da tecnologia de desenvolvimento
vai considerá frameworks multiplataforma para otimizar recursos e tempo de
desenvolvimento, no caso Flutter, O processo de desenvolvimento deverá seguir
ciclos iterativos curtos, permitindo ajustes contínuos baseados em feedback. As
funcionalidades essenciais implementadas no MVP incluirão: tela de cadastro e
autenticação de usuários; questionário digital baseado no modelo BFI para
captura de traços de personalidade; interface para visualização de pontos
turísticos recomendados com informações sobre acessibilidade (presença de
rampas, elevadores, banheiros adaptados, sinalização tátil); link para mapa
interativo mostrando localização dos pontos turísticos; sistema de filtros
considerando preferências pessoais e necessidades específicas de
acessibilidade; Criação de listas de pontos turísticos compartilháveis entre
usuários; e um sistema de avaliação colaborativo dos pontos turísticos. A
interface será projetada considerando os requisitos levantados na Fase 1,
implementando elementos como botões grandes com áreas de toque ampliadas,
fontes ajustáveis, alto contraste entre texto e fundo, navegação simplificada
com poucos níveis hierárquicos, feedback claro para cada ação do usuário, e
suporte a recursos nativos de acessibilidade do sistema operacional (VoiceOver
no iOS e TalkBack no Android). A prototipação vai ser realizada utilizando
ferramentas de design como Figma, permitindo validação visual antes da
codificação.

\subsubsection{Fase 4: Avaliação da Eficácia do Aplicativo através de Testes com Usuários}
A fase final foca na validação do aplicativo através de testes de usabilidade e
aceitação com diferentes perfis de usuários. A amostra deve ser selecionada por
conveniência, incluindo: pessoas idosas com diferentes níveis de familiaridade
com tecnologia; pessoas com deficiência física ou mobilidade reduzida; e
profissionais de turismo e acessibilidade que atuam como validadores técnicos.
Os testes devem ser realizados em ambiente controlado, permitindo observação
direta da interação dos usuários com o aplicativo, onde deve ser observado a
execução de tarefas pré-definidas, como: realizar cadastro no aplicativo;
responder ao questionário de personalidade; visualizar recomendações de pontos
turísticos; filtrar pontos por critérios de acessibilidade; e observar
comentários de outros usuários sobre o ponto turístico em evidência. Os
resultados das sessões devem ser registrados para posterior analise da eficácia
do sistema. A avaliação da precisão das recomendações envolverá análise
comparativa entre as sugestões geradas pelo sistema e as preferências reais
declaradas pelos participantes em questionário pós-teste, calculando métricas
de acurácia e relevância. A adequação às condições de acessibilidade será
verificada confrontando as informações apresentadas pelo aplicativo com dados
reais dos estabelecimentos turísticos, obtidos através de fontes de dados
seguras. Considerações Éticas Todos os procedimentos envolvendo participantes
humanos devem ser conduzidos respeitando os princípios éticos de pesquisa,
conforme Resolução CNS 466/2012. Os participantes receberam Termo de
Consentimento Livre e Esclarecido (TCLE) detalhando objetivos, procedimentos,
riscos mínimos e benefícios da pesquisa, com garantia de confidencialidade e
possibilidade de retirada a qualquer momento sem prejuízos. O projeto deverá
ser submetido à apreciação do Comitê de Ética em Pesquisa institucional.
Análise de Dados Os dados quantitativos devem ser analisados através de
estatística descritiva utilizando ferramentas como softwares em R ou Python. Os
dados qualitativos (transcrições de entrevistas, observações dos testes de
usabilidade) foram submetidos à análise de conteúdo temática, identificando
padrões recorrentes, dificuldades comuns e sugestões de melhorias. Atendendo
simultaneamente aos requisitos técnicos de integração com sistemas de
recomendação inteligentes e às necessidades reais de acessibilidade e
usabilidade do público-alvo, contribuindo tanto para o avanço acadêmico quanto
para a inclusão social no turismo.

\begin{figure}[H]
    \centering
    %    \includegraphics[width=0.85\textwidth]{fluxo_metodologia.png}
    \caption{Fluxo metodológico proposto.}
\end{figure}

% ---------------------------------------------------------------
% 4 CRONOGRAMA
% ---------------------------------------------------------------
\newpage
\section{Cronograma}
\begin{table}[H]
    \centering
    \begin{tabular}{|p{6cm}|c|c|c|c|c|c|}
        \hline
        \textbf{Atividade}             & \textbf{Nov./25} & \textbf{Dez./25} & \textbf{Jan./26} & \textbf{Fev./26} & \textbf{Mar./26} & \textbf{Abr./26} \\
        \hline
        Revisão de Referencial Teórico & X                & X                &                  &                  &                  &                  \\
        \hline
        Modelagem do sistema           &                  & X                &                  &                  &                  &                  \\
        \hline
        Criação de endpoints da API    &                  &                  & X                &                  &                  &                  \\
        \hline
        Criação de Protótipo           &                  &                  &                  & X                &                  &                  \\
        \hline
        Criação de telas do App        &                  &                  &                  & X                & X                &                  \\
        \hline
        Integração do App com a API    &                  &                  &                  & X                & X                &                  \\
        \hline
        Testes e validação             &                  &                  &                  &                  & X                & X                \\
        \hline
    \end{tabular}
    \caption{Cronograma de execução do projeto.}
\end{table}

% ---------------------------------------------------------------
% 5 REFERÊNCIAS
% ---------------------------------------------------------------
\newpage
\section{Referências}
\begin{singlespace}
    \setlength\bibitemsep{10pt}
    \sloppy
    %\nocite{*}
    \printbibliography[heading=none]
\end{singlespace}
\end{document}

