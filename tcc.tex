\documentclass[12pt,a4paper]{article}

% ---------------------------------------------------------------
% PACOTES BÁSICOS E FORMATAÇÃO ABNT
% ---------------------------------------------------------------
\usepackage[brazil]{babel}
\usepackage[utf8]{inputenc}
\usepackage[T1]{fontenc}
\usepackage[a4paper,top=3cm,bottom=2cm,left=3cm,right=2cm]{geometry}
\usepackage{setspace}
\usepackage{times}
\usepackage{indentfirst}
\usepackage{graphicx}
\usepackage{float}
\usepackage{hyperref}
\usepackage{enumitem}
\usepackage{titlesec}
\usepackage{ragged2e}
\usepackage{caption}
\usepackage[
  style=abnt,
  uniquename=init,
  giveninits,
  uniquelist=true,
  maxbibnames=4,
  repeatfields=true,
  justify
]{biblatex}
\addbibresource{bibliografia.bib}

% ---------------------------------------------------------------
% REGRAS DE FORMATAÇÃO
% ---------------------------------------------------------------
\setlength{\parindent}{1.25cm}
\setlength{\parskip}{0pt}
\onehalfspacing % espaçamento 1,5 conforme ABNT

% Estilo das seções ABNT (em maiúsculas e negrito)
\titleformat{\section}{\bfseries\normalsize\uppercase}{\thesection.}{0.5em}{}
\titleformat{\subsection}{\bfseries\normalsize}{\thesubsection.}{0.5em}{}
\titleformat{\subsubsection}{\normalsize}{\thesubsubsection.}{0.5em}{}

% Espaçamento entre título e texto
\titlespacing*{\section}{0pt}{1.5em}{1em}
\titlespacing*{\subsection}{0pt}{1em}{0.5em}

% Legendas em formato ABNT
\captionsetup{font=small,labelfont=bf,labelsep=endash}

% ---------------------------------------------------------------
% CAPA
% ---------------------------------------------------------------
\begin{document}
\thispagestyle{empty}
\begin{center}
    \begin{center}
        \includegraphics{logotipos/DcompIfma.png}
    \end{center}
    \singlespacing
    \textbf{INSTITUTO FEDERAL DE EDUCAÇÃO, CIÊNCIA E TECNOLOGIA DO MARANHÃO}\\
    \textbf{CAMPUS SÃO LUÍS - MONTE CASTELO}\\
    \textbf{DEPARTAMENTO DE COMPUTAÇÃO}\\
    \textbf{CURSO DE BACHARELADO EM SISTEMAS DE INFORMAÇÃO}\\[4cm]

    \textbf{PAULO HENRIQUE SILVA PINHEIRO}\\[4cm]

    \textbf{\MakeUppercase{Desenvolvimento de um aplicativo móvel integrado a um sistema de recomendação para turismo acessível com foco na pessoa idosa: Estudo de caso do Projeto Persona Tour}}\\[4cm]

    \vfill
    \textbf{SÃO LUÍS – MA}\\
    \textbf{2025}
\end{center}

% ---------------------------------------------------------------
% FOLHA DE ROSTO
% ---------------------------------------------------------------
\newpage
\thispagestyle{empty}
\begin{center}
    \textbf{PAULO HENRIQUE SILVA PINHEIRO}\\[3cm]

    \textbf{\MakeUppercase{Desenvolvimento de um aplicativo mobile integrando um sistema de recomendação para turismo inclusivo com foco na pessoa idosa: Estudo de caso do Projeto Persona Tour}}\\[2cm]
\end{center}

\begin{flushright}
    \begin{minipage}{0.6\textwidth}
        \justifying
        \noindent
        Projeto de Trabalho de Conclusão de Curso apresentado à disciplina \textbf{Monografia I}, ministrada pelo Prof. Dr. Hélder Pereira Borges, como requisito básico para o Trabalho de Conclusão de Curso.\\[0.25cm]

        \noindent
        Orientadora: Profa. Dra. Eveline de Jesus Viana Sá
    \end{minipage}
\end{flushright}

\vfill
\begin{center}
    \textbf{SÃO LUÍS – MA}\\
    \textbf{2025}
\end{center}

% ---------------------------------------------------------------
% RESUMO
% ---------------------------------------------------------------
\newpage
\thispagestyle{empty}
\section*{RESUMO}
\begin{singlespace} % espaçamento simples conforme NBR 6028
    \justifying
    Este trabalho apresenta o desenvolvimento de um aplicativo mobile de recomendação de ponto
    turísticos com foco em acessibilidade para pessoas idosas e com deficiência em São Luís - MA. O objetivo é oferecer recomendações personalizadas adaptadas às necessidades específicas
    dos usuários, promovendo inclusão social e digital no contexto turístico. A justificativa
    fundamenta-se no dado alarmante de que 53,5\% dos turistas com deficiência deixam de
    viajar no Brasil por falta de acessibilidade, e na ausência de soluções que integrem
    personalidade com acessibilidade como elemento central. O diferencial da proposta
    reside na integração simultânea do modelo psicológico Big Five Inventory (BFI)
    com restrições de acessibilidade, gerando recomendações que respeitam tanto
    preferências individuais quanto garantem viabilidade física e segurança da experiência.
    A metodologia segue Design Science Research com quatro fases: investigação de técnicas
    de interface para idosos via revisão sistemática e pesquisa participativa;
    desenvolvimento de uma API REST em Java Spring Boot para lógica de negócios do
    aplicativo e integração com a API do sistema de recomendação de pontos turísticos;\@
    implementação de um MVP do aplicativo móvel com funcionalidades
    de cadastro, questionário BFI, visualização de recomendações com indicadores de
    acessibilidade e sistema colaborativo de avaliação de pontos turísticos; validação
    através de testes de usabilidade com usuários reais. Espera-se contribuir com este projeto o fortalecimento
    da inclusão digital de pessoas idosas e com deficiência, demonstrando viabilidade de
    sistemas inteligentes que combinam algoritmos de recomendação com design centrado no usuário,
    alinhado com a Lei Brasileira de Inclusão (Lei 13.146/2015).

    \bigskip
    \noindent\textbf{Palavras-chave:} Aplicação móvel. Sistema de recomendação. Idoso. Turismo inclusivo. Acessibilidade.
\end{singlespace}

\newpage
\thispagestyle{empty}
\section*{Abstract}
\begin{singlespace} % espaçamento simples conforme NBR 6028
    \justifying
    This work presents the development of a mobile application for recommending tourist attractions with a focus on accessibility for elderly people and people with disabilities in São Luís - MA. The objective is to offer personalized recommendations adapted to the specific needs of users, promoting social and digital inclusion in the tourism context. The justification is based on the alarming fact that 53.5\% of tourists with disabilities fail to travel in Brazil due to lack of accessibility, and the absence of solutions that integrate personality with accessibility as a central element. The differential of the proposal lies in the simultaneous integration of the Big Five Inventory (BFI) psychological model with accessibility restrictions, generating recommendations that respect both individual preferences and guarantee physical feasibility and safety of the experience. The methodology follows Design Science Research with four phases: investigation of interface techniques for the elderly via systematic review and participatory research; Development of a REST API in Java Spring Boot for the application's business logic and integration with the API of the tourist attraction recommendation system; Implementation of an MVP of the mobile application with functionalities such as registration, BFI questionnaire, visualization of recommendations with accessibility indicators and a collaborative system for evaluating tourist attractions; validation through usability tests with real users. This project is expected to contribute to strengthening the digital inclusion of elderly and disabled people, demonstrating the viability of intelligent systems that combine recommendation algorithms with user-centered design, aligned with the Brazilian Inclusion Law (Law 13.146/2015).

    \bigskip
    \noindent\textbf{Keywords:} Mobile Application. Recommendation System. Elderly. Inclusive Tourism.
    Accessibility.

\end{singlespace}

% ---------------------------------------------------------------
% SUMÁRIO
% ---------------------------------------------------------------
\newpage
\thispagestyle{empty}
\tableofcontents
\newpage

% ---------------------------------------------------------------
% 1 INTRODUÇÃO
% ---------------------------------------------------------------
\newpage
\section{Introdução}
\justifying

Sistemas de recomendação em aplicativos móveis têm desempenhado um papel cada vez mais relevante em diferentes setores, especialmente no turismo, onde proporcionam experiências personalizadas e facilitam a descoberta de serviços e destinos alinhados aos interesses e necessidades dos usuários. Tais sistemas utilizam algoritmos inteligentes para analisar dados comportamentais, preferências explícitas e informações contextuais, sugerindo pontos turísticos, restaurantes, eventos e outras atrações de acordo com o perfil de cada pessoa. A crescente popularização de smartphones consolidou os aplicativos de turismo como ferramentas essenciais, permitindo que recomendações em tempo real sejam integradas ao planejamento de viagens e roteiros personalizados.

Com o avanço dessas tecnologias, destaca-se a importância de incorporar critérios de acessibilidade nas recomendações, sobretudo para atender públicos com necessidades específicas, como idosos e pessoas com deficiência. A personalização baseada em acessibilidade envolve avaliar obstáculos arquitetônicos, disponibilidade de serviços especializados, informações sobre o acesso, e demais fatores que interferem na experiência do turista com mobilidade reduzida ou outras limitações temporárias ou permanentes. Aplicar tais critérios em sistemas de recomendação amplia a inclusão, tornando o turismo mais democrático, seguro e satisfatório para todos.

Em 2023 teve início o projeto de pós-doutorado intitulado "Uma análise do uso de IA na geração de rotas turísticas personalizadas para o público idoso: um estudo de caso no Projeto The Route", desenvolvido no Instituto Superior de Engenharia do Porto (ISEP/IPP), em Portugal. O Projeto The Route \cite{faria2019theroute} evoluiu para o GrouPlanner \cite{alves2022grouplanner}, coordenado pelo GECAD/ISEP-IPP, incorporando recomendações de roteiros para grupos baseadas nos traços de personalidade do modelo Big Five \cite{Costa1992} e técnicas de negociação para lidar com heterogeneidade grupal na região do Porto e norte de Portugal \cite{alves2023group}.

Inspirado nesse contexto e na evolução das pesquisas, estudou-se a viabilidade de aplicar tais conceitos ao turismo acessível no Brasil, com ênfase na população idosa. Assim, um projeto de iniciação científica foi desenvolvido no Instituto Federal do Maranhão (IFMA) Campus São Luís - Monte Castelo entre agosto de 2024 e agosto de 2025, intitulado "Um estudo sobre os algoritmos inteligentes utilizados no Sistema de Recomendação para Grupos aplicado ao Turismo: um estudo de caso do Projeto GrouPlanner". O projeto resultou na implementação de um sistema de recomendação o qual foi nomeado Pesona Tour, baseado no algoritmo d-means \cite{Alves2024}, adaptado para considerar aspectos de acessibilidade em pontos turísticos na cidade de São Luís - MA. O sistema utiliza o modelo BFI para agrupar usuários com perfis psicológicos semelhantes, recomendando pontos turísticos que atendam às suas preferências e necessidades específicas de acessibilidade. Com base nos resultados positivos obtidos nesse projeto inicial, o presente trabalho propõe a continuidade e adaptação desses avanços tecnológicos por meio do desenvolvimento do aplicativo para o serviço Persona Tour.

O objetivo central consiste em desenvolver uma aplicação móvel voltada ao turismo acessível que faz uso do sistema de recomendação turística para obtenção dos pontos turísticos, com especial atenção às necessidades do público idoso. O aplicativo visa oferecer uma experiência turística personalizada, acessível e sensível às limitações e preferências do usuário, utilizando o modelo psicológico BFI (Big Five Inventory – BFI) e uma rede colaborativa dentro do App, reforçando os recursos disponíveis em pontos turísticos.


\newpage
% ---------------------------------------------------------------
% 1.1 PROBLEMA
% ---------------------------------------------------------------
% \newpage
\subsection{Problema}
\justifying
A acessibilidade é um fator decisivo para a efetiva inclusão no
turismo \cite{unwto2016manual}. Para turistas com deficiência
ou mobilidade reduzida, não basta apenas saber “o que visitar”,
mas se os locais dispõem de recursos como rampas, banheiros
adaptados, piso tátil, sinalização adequada ou rotas sem degraus,
informações que raramente aparecem nos catálogos disponíveis de
forma confiável e comparável. Essa lacuna gera barreiras à autonomia
dos viajantes e reduz o interesse em locais que não evidenciam informações
sobre acessibilidade. Ainda assim, o tema da acessibilidadepermanece como um dos
aspectos menos contemplados pela maioria dos sistemas de
recomendação turísticos atuais \cite{santos2019poiaccessibility}.
Nesse sentido, a personalização aliada à acessibilidade
torna-se essencial para melhorar a experiência turística e ampliar a
competitividade econômica do setor \cite{Silva2023}

Portanto, esta pesquisa se propõe a responder à seguinte indagação central:
Como elaborar uma aplicação capaz de satisfazer as necessidades de pessoas com
deficiência e idosos permitindo o acesso a informação de pontos turisitcos
baseados em suas preferências e condições e promovendo a devida inclusão
social, turística e digital?.

% ---------------------------------------------------------------
% 1.2 JUSTIFICATIVA
% ---------------------------------------------------------------
\newpage
\subsection{Justificativa}
\justifying
Avançando nesse caminho, a presente proposta insere a acessibilidade como eixo
central da personalização,
transformando informações sobre barreiras e recursos adaptadas para recomendação de
pontos turísticos coesos com o usuário. Assim,
não apenas pontos turísticos são sugeridos, mas condições de acessibilidade do
local e preferências individuais do usuário e do grupo.
Com isso, o projeto justifica-se pela necessidade de superar as limitações dos
recomendadores tradicionais, que priorizam popularidade e cliques,
mas desconsideram variáveis críticas de inclusão \cite{Azambuja2021}.
Social e legalmente, a proposta materializa princípios da Lei Brasileira de
Inclusão ao trazer a acessibilidade para o centro do processo decisório
, academicamente, contribui para a inovação em sistemas de necessidade
social \cite{brasil2014turismoacessivel} \cite{brasil2015lei13146}.

Diante desse contexto, a proposta do Persona Tour justifica-se pela necessidade
de promover a inclusão digital e social da pessoa idosa e da pessoa com
deficiência, no turismo, utilizando a tecnologia como mediadora de experiências
significativas, seguras e adaptadas às suas limitações e interesses. O
aplicativo busca não apenas recomendar destinos e pontos de interesse, mas
também compreender as preferências e necessidades de cada usuário, propondo
lugares compatíveis com suas motivações, nível de atividade e preferências
categóricas. Além disso, a interface do aplicativo foca em ser acessível, com
elementos visuais, tipográficos e interativos adaptados à ergonomia e à
usabilidade de pessoas idosas, contribuindo para a redução das barreiras de
acesso à informação turística.

% ---------------------------------------------------------------
% 1.3 OBJETIVOS
% ---------------------------------------------------------------
\newpage
\subsection{Objetivos}
\subsubsection{Objetivo Geral}
Desenvolver um aplicativo móvel de recomendação de pontos turísticos,
delimitado na recomendação de locais da cidade de São Luís, com foco na
acessibilidade pessoas com deficiência e idosos.

\subsubsection{Objetivos Específicos}
\begin{itemize}
    \item[a.] Investigar técnicas de interface intuitiva para os usuários idosos;
    \item[b.] Desenvolver uma API para integrar o sistema de recomendação ao aplicativo
          móvel;
    \item[c.] Implementar um mínimo produto viável do aplicativo móvel;
    \item[d.] Avaliar a eficácia do aplicativo por meio de testes com diferentes perfis
          de usuários, verificando a precisão das recomendações, a adequação às condições
          de acessibilidade e a percepção de usabilidade;
\end{itemize}

% ---------------------------------------------------------------
% 2 REFERENCIAL TEÓRICO
% ---------------------------------------------------------------
\newpage
\section{Referencial Teórico}
\justifying
\subsection{Turismo Acessível e Acessibilidade}

O turismo representa uma das atividades econômicas e sociais mais importantes
para países em desenvolvimento, funcionando como gerador de renda, emprego e
intercâmbio cultural. Contudo, a participação igualitária nessa prática depende
fundamentalmente da existência de infraestrutura e serviços adequados que
considerem as necessidades específicas de pessoas com deficiência e mobilidade
reduzida, incluindo idosos. A acessibilidade turística transcende a
disponibilidade de rampas e banheiros adaptados; ela envolve um conceito mais
amplo que contempla elementos informativos, comunicacionais, tecnológicos e
atitudinais.

Segundo dados recentes do Ministério do Turismo, mais de 53,5\% dos turistas
com deficiência deixaram de viajar para algum destino no Brasil por falta de
acessibilidade, revelando uma exclusão sistemática desse
público~\cite{MTur2024}. O Brasil, embora reconhecido internacionalmente por
suas belezas naturais, apresenta graves deficiências na operacionalização de
políticas de turismo inclusivo, conforme demonstrado em análises de políticas
públicas dos últimos 20 anos~\cite{Vilela2022}.

Um caso emblemático dessa falha institucional é o ``Programa Turismo
Acessível'', projeto desenvolvido pelo próprio governo brasileiro para mapear
acessibilidade em pontos turísticos. O aplicativo, embora bem-intencionado,
apresenta limitações críticas: dados desatualizados, informações inconsistentes
sobre acessibilidade real dos espaços, ausência de validação contínua pelas
comunidades de pessoas com deficiência, e interface pouco acessível para o
público idoso. Muitos pontos turísticos cadastrados como acessíveis carecem de
infraestrutura essencial como rampas inadequadas, banheiros sem adaptações e
falta de sinalização tátil, demonstrando que a presença de informação não
garante qualidade de implementação.

A Lei Brasileira de Inclusão da Pessoa com Deficiência (Lei 13.146/2015) e o
Manual de Turismo Acessível da UNWTO estabelecem que acessibilidade é direito
fundamental, não benefício~\cite{Brasil2015, UNWTO2016}. Entretanto, a
implementação permanece lenta e desigual. O conceito de turismo acessível deve
ser compreendido em múltiplas dimensões complementares. A acessibilidade
arquitetônica refere-se à remoção de barreiras físicas, incluindo rampas
adequadas com inclinação máxima de 8,33\%, corrimões, elevadores, pisos táteis
e banheiros com espaço de manobra mínimo de 1,5m de diâmetro. A acessibilidade
comunicacional envolve a disponibilidade de informações em formatos
alternativos como Braille, áudio, linguagem simples e descrição de imagens,
aliada a sinalização clara e legível com descrição de ambientes. A
acessibilidade atitudinal compreende o treinamento de profissionais de turismo
para atender com qualidade e respeito, promovendo a compreensão da
neurodiversidade e das diferentes formas de deficiência, valorizando a
autonomia do visitante. Por fim, a acessibilidade tecnológica garante a
disponibilidade de recursos assistivos como leitores de tela, amplificadores e
sistemas de amplificação de som, além da integração com tecnologias móveis e
interfaces acessíveis em plataformas digitais.

\subsection{Sistemas de Recomendação no Turismo}

Os Sistemas de Recomendação (SR) têm revolucionado a forma como os indivíduos
descobrem produtos, serviços e informações em ambientes digitais. No contexto
turístico, eles auxiliam viajantes a identificar pontos de interesse relevantes
em meio à sobrecarga informacional, reduzindo significativamente o tempo de
pesquisa e aumentando a qualidade da experiência~\cite{Resnick1997,
    Santos2023}. Um sistema de recomendação funciona como intermediário entre a
diversidade de ofertas turísticas e as preferências específicas de cada
usuário, processando dados comportamentais, demográficos e psicológicos para
gerar sugestões personalizadas.

As abordagens tradicionais em Sistemas de Recomendação Turística baseiam-se
predominantemente em duas técnicas: a filtragem colaborativa, que identifica
usuários com preferências similares e recomenda itens apreciados por esses
``vizinhos''; e a filtragem baseada em conteúdo, que analisa características
dos pontos turísticos como categoria, localização e características,
comparando-as com o perfil do usuário~\cite{Resnick1997}. Contudo, essas
abordagens possuem limitações reconhecidas. A filtragem colaborativa sofre com
o problema de ``partida a frio'' caracterizado por poucos dados históricos de
usuários novos, enquanto a filtragem por conteúdo tende a criar ``bolhas de
informação'' que reduzem a serendipidade e a descoberta de novas experiências.

O \textbf{GrouPlanner}~\cite{Alves2022, Alves2023}, desenvolvido pelo
GECAD/ISEP-IPP (Grupo de Pesquisa em Engenharia Computacional de Sistemas
Adaptativos Complexos do Instituto Superior de Engenharia do Porto), representa
um avanço significativo ao incorporar modelagem de personalidade como base para
recomendação. O GrouPlanner utiliza o modelo psicológico \textbf{Big Five
    Inventory (BFI)}~\cite{Costa1992} para capturar os cinco principais traços de
personalidade humana. A abertura para experiências reflete a propensão para
novelidade, criatividade e flexibilidade, influenciando a busca por atrações
inovadoras, culturais e desafiadoras. A conscienciosidade relaciona-se com
organização, planejamento e atenção a detalhes, influenciando a preferência por
roteiros estruturados e segurança. A extroversão determina a sociabilidade e
busca por estimulação, afetando a preferência por atividades em grupo, vida
noturna ou contemplação solitária. A agradabilidade representa a orientação
para cooperação e harmonia, afetando a disposição para turismo comunitário e
experiências inclusivas. Por fim, o neuroticismo refere-se à tendência para
emoções negativas, relacionando-se com sensibilidade a stress ambiental,
necessidade de conforto e segurança.

O BFI é implementado através de um questionário validado
cientificamente~\cite{Costa1992, Andrade2023} que permite classificar usuários
em um espaço multidimensional de personalidade. A principal inovação do
GrouPlanner reside na aplicação do algoritmo \textit{d-means}~\cite{Alves2024}
para agrupar usuários com perfis psicológicos similares e gerar recomendações
coletivas em contextos de grupo, negociando preferências heterogêneas de forma
harmoniosa.

Para o Persona Tour, a arquitetura de recomendação foi adaptada do GrouPlanner,
incorporando uma dimensão crítica ausente em sistemas tradicionais: a modelagem
de restrições de acessibilidade. O sistema processa não apenas traços de
personalidade, mas também informações sobre o tipo e grau de deficiência ou
mobilidade reduzida, necessidades específicas como presença de elevador, piso
tátil, intérprete de Libras e descrição de ambientes, além de preferências de
tipo de turismo ajustadas às capacidades do usuário. A integração entre BFI e
restrições de acessibilidade cria um modelo de recomendação verdadeiramente
inclusivo, que não apenas respeita preferências de personalidade mas também
garante a viabilidade física e segurança da experiência turística. Esse
diferencial posiciona o Persona Tour como inovação no campo do turismo
inteligente (Tourism 4.0), alinhado com políticas públicas de inclusão
social~\cite{Silva2023}.

\subsection{Aplicações Mobile e Experiência do Usuário (UX)}

A proliferação de dispositivos móveis transformou a forma como as pessoas
interagem com informações, serviços e experiências. Aplicativos móveis
representam 90\% do tempo de engajamento digital, tornando-os o canal
preferencial para acesso a serviços turísticos. Contudo, a qualidade e
acessibilidade dessas aplicações variam enormemente, com muitas falhando em
atender necessidades de usuários com deficiência ou idosos~\cite{Bevan2021,
    Bruffaerts2021}.

\subsubsection{Acessibilidade em Aplicações Móveis: WCAG 2.1}

As \textbf{Web Content Accessibility Guidelines 2.1 (WCAG 2.1)}~\cite{W3C2021},
desenvolvidas pelo World Wide Web Consortium, estabelecem padrões
internacionais reconhecidos para acessibilidade digital. Embora originalmente
focadas em conteúdo web, suas diretrizes são amplamente aplicáveis a aplicações
móveis, especialmente em nível AA que garante acessibilidade para 80\% das
necessidades de pessoas com deficiência~\cite{W3C2021, MDPI2021}.

Os princípios WCAG 2.1 fundamentais para aplicações móveis dividem-se em quatro
categorias. A perceptibilidade exige que a informação seja apresentada de forma
compreensível, especialmente através de contraste de cores com razão mínima de
4,5:1 entre texto e fundo, ou 3:1 para texto grande com tamanho igual ou
superior a 18 pontos. Aplicações frequentemente violam esse requisito, causando
ilegibilidade para pessoas com baixa visão. Adicionalmente, o tamanho de texto
deve ser ajustável e a interface deve permanecer utilizável em escalas até
200\%, com descrição de elementos visuais como botões, ícones e imagens que
devem possuir rótulos textuais ou descrições acessíveis para leitores de tela.

A operabilidade garante que todos os componentes sejam operáveis sem exclusão
de usuários com diferentes capacidades motoras ou cognitivas. O tamanho de alvo
tátil é crítico, exigindo que botões e elementos interativos possuam área
mínima de 48x48 pixels lógicos para facilitar ativação por usuários com
limitações motoras finas~\cite{Flutter2025, Dev2023}. Toda funcionalidade deve
ser acessível via teclado, suportando navegação com abas e foco claramente
visível. Leitores de tela como TalkBack para Android e VoiceOver para iOS devem
descrever a interface de forma coerente e permitir navegação completa. Além
disso, mudanças de contexto automáticas devem ser evitadas, pois alterações não
devem ocorrer sem confirmação do usuário, especialmente durante entrada de
dados.

A compreensibilidade exige que interface, comportamento e mensagens sejam
claros e antecipáveis. Isso envolve o uso de linguagem simples com texto
legível e vocabulário apropriado, sendo especialmente importante para usuários
idosos e com deficiência cognitiva~\cite{Recomendacoes2015}. As ações do
usuário devem gerar feedback imediato através de feedback visual, auditivo ou
tátil confirmando sucesso ou indicando erro com sugestão de correção. Padrões,
cores e posicionamento devem permanecer consistentes ao longo da aplicação para
facilitar a navegação intuitiva.

A robustez garante que o aplicativo funcione com diversas tecnologias
assistivas e em diferentes contextos. A compatibilidade com tecnologias
assistivas é fundamental, exigindo funcionamento adequado com leitores de tela,
amplificadores de som e interfaces adaptadas. Comportamento consistente deve
ser garantido em smartphones, tablets e versões antigas e novas do sistema
operacional através de testes em múltiplos dispositivos.

\subsubsection{Recomendações Específicas para Idosos}

Estudos sobre interação de idosos com aplicações
móveis~~\cite{Recomendacoes2015, LUSUI2021, Vilela2022} identificam desafios
cognitivos, visuais e motores específicos que exigem diretrizes adaptadas. A
degradação visual natural caracteriza-se pela redução de acuidade visual,
dificuldade com contraste e aumento de sensibilidade a cintilação,
recomendando-se font-size mínimo de 16-18 pontos, espaçamento entre elementos
ampliado e aumento de espaçamento de linha para melhor legibilidade.

A redução de destreza motora associada ao tremor, artrite e redução de precisão
manual exige que botões sejam significativamente maiores que 48x48 pixels, com
recomendação de 60-72 pixels, mantendo espaçamento amplo entre
elementos~\cite{Recomendacoes2015}. A mudança cognitiva caracteriza-se pelo
aumento de tempo de processamento, dificuldade com abstrações visuais e
tendência à familiaridade com padrões conhecidos. Nesses casos, navegação deve
ser simplificada com máximo de 3 níveis hierárquicos, nomes de botões claros e
descritivos, evitando ícones sem rótulo que possam causar confusão.

A proficiência tecnológica variável entre idosos, onde muitos possuem
experiência limitada com aplicativos móveis, exige que se priorize
familiaridade através de padrões conhecidos como enviar/receber para
comunicação, tutoriais integrados e suporte à ajuda contextual. O design deve
ser construído com a premissa de que o usuário idoso necessita de uma curva de
aprendizado reduzida e de máxima clareza em cada interação.

\subsubsection{Experiência do Usuário (UX) Centrada no Usuário Específico}

A Experiência do Usuário vai além da mera funcionalidade; refere-se ao conjunto
de emoções, atitudes e comportamentos associados ao uso de um
sistema~\cite{Nielsen2012}. Para públicos com necessidades especiais, UX deve
ser construída através de pesquisa participativa, onde o usuário é co-designer
do sistema.

O \textbf{Design Thinking} propõe que aplicações sejam desenvolvidas em
iterações sucessivas. A fase de descoberta busca entender profundamente as
necessidades reais dos usuários. A definição reformula o problema com base em
insights gerados. A ideação gera múltiplas soluções para o problema
identificado. A prototipação constrói versões tangíveis que permitem visualizar
a solução. O teste valida com usuários reais se a solução atende às
necessidades.

Para aplicações turísticas, isso significa envolver desde o início pessoas
idosas e com deficiência em sessões de prototipagem e teste de usabilidade,
capturando não apenas erros funcionais, mas também a satisfação emocional e o
senso de autonomia gerado pela experiência. A participação ativa desses
usuários garante que a aplicação seja realmente acessível e útil, não apenas em
conformidade técnica com diretrizes, mas em experiência prática e
significativa.

\subsection{Tecnologias e Arquitetura do Sistema}

O Projeto é fundamentado em uma arquitetura de três camadas que separa
responsabilidades e permite escalabilidade. Essa separação arquitetônica
possibilita que cada camada seja desenvolvida, mantida e evoluída
independentemente, facilitando a incorporação de novas tecnologias, algoritmos
e funcionalidades sem impactar as demais camadas.

\subsubsection{Camada 1: API de Sistema de Recomendação --- FastAPI Python}

A primeira camada implementa o sistema central de recomendação, desenvolvida em
FastAPI Python. Essa API é responsável por processar perfis de personalidade do
usuário e gerar recomendações de pontos turísticos através do algoritmo
d-means, consumindo dados da Google Maps API como terceiro para enriquecer as
recomendações com informações geográficas. O algoritmo d-means recebe como
entrada o perfil BFI do usuário representado por cinco dimensões de
personalidade e suas restrições específicas de acessibilidade, agrupando pontos
turísticos que combinam características que atraem usuários com aquele perfil
psicológico enquanto garantem conformidade com requisitos específicos de
acessibilidade. A integração com Google Maps API permite consumir dados
geográficos, localização, categoria, avaliações, horários e contato dos pontos
turísticos, enriquecendo essas informações com metadados de acessibilidade
provenientes do banco de dados do sistema. O processamento inteligente aplica
lógica sofisticada de matching entre perfil de personalidade e características
dos pontos turísticos, retornando lista ordenada de recomendações com score de
relevância que reflete o grau de correspondência. A validação de dados de
acessibilidade garante que as recomendações atendem rigorosamente aos
requisitos específicos de mobilidade, visão e audição declarados pelo usuário,
evitando sugerir locais que, embora alinhados com a personalidade, possam ser
inacessíveis. Essa camada funciona como motor de inteligência, independente de
tecnologias de frontend ou regras de negócio específicas do aplicativo, podendo
ser facilmente integrada em diferentes contextos.

\subsubsection{Camada 2: API de Negócios e Gestão de Dados --- Java Spring Boot}

A segunda camada implementa toda a lógica de negócio do aplicativo,
desenvolvida em Java Spring Boot com banco de dados PostgreSQL. Essa API
centraliza múltiplas responsabilidades críticas para o funcionamento da
plataforma. O módulo de autenticação e perfil gerencia cadastro de usuários com
armazenamento seguro de credenciais através de criptografia avançada, validação
de email para verificação de contas e persistência de dados de perfil no banco
de dados PostgreSQL. O módulo de questionário BFI implementa o Big Five
Inventory validado em português brasileiro, composto por quinze itens na versão
reduzida, coletando respostas em escala Likert de cinco pontos e calculando
automaticamente scores para cada dimensão de personalidade. O módulo de
integração com sistema de recomendação orquestra a comunicação entre os
componentes da plataforma, capturando o perfil BFI calculado do usuário,
coletando suas restrições específicas de acessibilidade, enviando requisição
devidamente formatada à FastAPI Python para processar recomendações, recebendo
a lista de pontos turísticos recomendados e enriquecendo a resposta com dados
complementares armazenados no banco local, como avaliações anteriores e
comentários de outros usuários.

O módulo de dados de acessibilidade armazena e gerencia metadados detalhados
sobre cada ponto turístico incluindo informações sobre rampas, elevadores,
banheiros adaptados, sinalização tátil e outros requisitos de acessibilidade,
permitindo buscas facetadas por tipo de acessibilidade que o usuário necessita.
O módulo de comentários e avaliações implementa um sistema colaborativo onde
usuários avaliam pontos turísticos, relatam experiências de acessibilidade,
compartilham fotos e comentários que enriquecem a informação disponível para
futuros visitantes. Esse sistema implementa validação por pares e moderação de
conteúdo inadequado para garantir qualidade e confiabilidade das informações. O
módulo de filtros e preferências gerencia preferências pessoais de categorias
turísticas, necessidades específicas de acessibilidade, histórico de visitas e
listas compartilháveis entre usuários, permitindo customização profunda da
experiência. O banco de dados PostgreSQL organiza tabelas estruturadas de
usuários, perfis BFI, restrições de acessibilidade, histórico de recomendações
e interações, sistema de avaliações e comentários colaborativos, além de cache
de dados de pontos turísticos sincronizados com informações da FastAPI Python.

Os requisitos de segurança são rigorosos, implementando autenticação JWT (JSON
Web Tokens) para validação de requisições, validação rigorosa de entrada para
prevenir ataques de injeção, CORS configurado para permitir apenas origens
autorizadas, criptografia de dados sensíveis em repouso utilizando AES-256, TLS
1.3 para comunicação criptografada em trânsito e conformidade total com LGPD
(Lei Geral de Proteção de Dados) através de auditoria contínua.

\subsubsection{Camada 3: Aplicação Móvel --- Flutter}

A terceira camada é a interface de usuário, implementada em Flutter, framework
open-source de desenvolvimento cross-platform da Google que permite criar
aplicações nativas em iOS e Android com código base único. A escolha do Flutter
justifica-se por múltiplas razões alinhadas aos objetivos de acessibilidade do
projeto. O framework possui suporte nativo a acessibilidade através de
integração built-in com tecnologias assistivas como TalkBack para Android e
VoiceOver para iOS, com widgets que fornecem propriedades como semanticLabel,
onTap e onLongPress para navegação fluida por leitor de tela. Os widgets do
Flutter são acessíveis por padrão, com componentes como Button, TextField e
AppBar já implementando acessibilidade mínima que desenvolvedores podem
estender. O framework oferece suporte a verificação automática de conformidade
com diretrizes WCAG 2.1 através de pacotes especializados. Flutter compila para
código nativo, oferecendo performance superior crucial para dispositivos com
recursos limitados comuns entre usuários idosos. O Hot Reload permite iteração
rápida durante desenvolvimento e teste de acessibilidade, acelerando o ciclo de
desenvolvimento e validação.

Os componentes principais da aplicação Flutter organizam-se em telas
especializadas. A tela de autenticação implementa login e registro com campos
amplos mantendo área mínima de 48 pixels, alta relação de contraste de 4,5:1,
feedback claro de erros indicando exatamente o que corrigir e suporte completo
a navegação por teclado. A tela de questionário BFI distribui a interface em
múltiplas páginas para evitar sobrecarga cognitiva do usuário, exibe indicador
de progresso claro mostrando quantas questões foram respondidas, oferece opção
de pausar e retomar posteriormente, e apresenta escala Likert com rótulos
textuais claros variando de ``Discordo Fortemente'' a ``Concordo Fortemente''.
A tela de recomendações exibe lista de pontos turísticos em card layout
acessível, com cada card mostrando foto com descrição alternativa para leitores
de tela, nome do local, categoria, distância até o ponto e indicadores visuais
de acessibilidade como ``Rampa disponível'', ``Banheiro adaptado'' e
``Estacionamento próximo''.

A tela de detalhes do ponto turístico fornece informação expandida incluindo
endereço completo, mapa interativo para visualizar localização, descrição
detalhada do espaço com audiodescrição de fotos disponível, informações
estruturadas de acessibilidade, contato do estabelecimento, horários de
funcionamento, sugestões de transporte acessível para chegar ao local,
avaliações de outros usuários sobre a acessibilidade e galeria de fotos. A tela
de avaliação e comentários permite que usuários avaliem a acessibilidade real
do ponto turístico, compartilhem experiências pessoais, relatem discrepâncias
entre informações cadastradas e realidade observada e participem de validação
colaborativa de informações. O perfil do usuário permite gerenciar informações
pessoais, restrições de acessibilidade, histórico de visitas anteriores, pontos
marcados como favoritos e configurações avançadas de acessibilidade incluindo
tamanho de fonte ajustável, modo de alto contraste e velocidade de animação.

A implementação de acessibilidade em Flutter segue rigorosamente as diretrizes
WCAG 2.1. A tipografia utiliza font-size mínimo de 16 pontos para corpo de
texto, 18-20 pontos para cabeçalhos, fonte sans-serif como Roboto ou Open Sans
para melhor legibilidade e suporte a aumento até 200\% para usuários com baixa
visão. Cores e contraste mantêm paleta com contraste mínimo de 4,5:1, modo de
alto contraste disponível como opção, e evita-se usar cor como único indicador
de informação, complementando com ícones, texto e padrões visuais. A navegação
estrutura-se hierarquicamente com máximo de três níveis para evitar confusão
cognitiva, botões ``Voltar'' conspícuos em cada página, breadcrumbs em páginas
profundas indicando o caminho percorrido e suporte completo a navegação por
teclado com ordem de foco lógica. Leitores de tela recebem suporte máximo com
todos os elementos interativos possuindo semanticLabel descritivo, imagens
tendo rótulos explicativos, ordem semântica otimizada para fluxo natural e
testes extensivos com TalkBack e VoiceOver.

Interações geram feedback multissensorial através de mudança visual de cor ou
ícone, feedback auditivo com som ou vibração do dispositivo e feedback textual
através de toast ou snackbar explicando resultado da ação. Campos obrigatórios
são claramente marcados com indicadores visuais, erros de validação aparecem
imediatamente com sugestão concreta de correção e ações importantes são
reversíveis, permitindo que usuários desfaçam operações sem medo de
consequências irreversíveis.

\subsubsection{Integração das Três Camadas}

\begin{figure}[H]
    \centering
    \IfFileExists{./figuras/arquitetura.pdf}{%
        \includegraphics[width=0.85\textwidth]{./figuras/arquitetura.pdf}%
    }{%
        \fbox{\begin{minipage}[c][0.3\textheight][c]{0.85\textwidth}\centering Arquivo de figura ausente: figuras/arquitetura.pdf \end{minipage}}%
    }
    \caption{Arquitetura do sistema. Fonte: Autor.}
\end{figure}

O fluxo de dados entre as camadas segue um padrão bem definido que garante
integração \textit{seamless}. O Flutter cliente autentica o usuário e captura
seu perfil BFI através do questionário interativo. Os dados são enviados para
Java Spring Boot API de negócios que processa o BFI, formata a requisição com
os parâmetros necessários e a envia à FastAPI Python. A FastAPI Python executa
o algoritmo d-means sofisticado, consulta Google Maps API para enriquecer dados
geográficos, e retorna lista de recomendações com scores de relevância.
Retornando à Java Spring Boot, enriquece a resposta com comentários, avaliações
e informações complementares do PostgreSQL respondidas por usuários
anteriormente, agregando conhecimento coletivo sobre acessibilidade.
Finalmente, Flutter exibe as recomendações personalizadas de forma visual e
acessível com informações claras de acessibilidade para cada ponto sugerido.

A segurança permeia toda a arquitetura, com dados sensíveis como localização e
informações de deficiência sendo criptografados em trânsito através de TLS 1.3
e em repouso utilizando AES-256. Usuários controlam explicitamente quais dados
compartilham e com quem, com consentimento informado e granular. A conformidade
com LGPD é garantida através de consentimento explícito para coleta de dados,
direito garantido de acesso aos dados pessoais, direito de exclusão de dados a
qualquer momento e auditorias regulares de segurança com testes de penetração
para identificar vulnerabilidades. Essa arquitetura em três camadas garante
separação rigorosa de responsabilidades, escalabilidade para suportar
crescimento de usuários e dados, manutenibilidade com código organizado e
testável, e permite que melhorias em uma camada beneficiem imediatamente toda a
plataforma sem necessidade de modificações complexas nas demais camadas.

\textbf{Segurança e Privacidade:}

\begin{itemize}
    \item Dados sensíveis (localização, informações de deficiência) criptografados em
          trânsito (TLS 1.3) e em repouso (AES-256)
    \item Usuários controlam quais dados compartilham
    \item Conformidade com LGPD através de consentimento explícito, direito de
          acesso/exclusão
    \item Auditorias regulares de segurança e testes de penetração
\end{itemize}

Essa arquitetura em três camadas garante \textbf{separação de
    responsabilidades}, \textbf{escalabilidade}, \textbf{manutenibilidade}, e
permite que melhorias em uma camada beneficiem imediatamente toda a plataforma.
A \textbf{FastAPI Python} concentra toda a inteligência do sistema de
recomendação, enquanto a \textbf{Java Spring Boot} gerencia todas as regras de
negócio do aplicativo e interação com usuários, e o \textbf{Flutter} fornece a
interface otimizada para acessibilidade do público-alvo.

\newpage

\section{Metodologia}

\justifying

\subsection{DSR - Design Science Research}

Este trabalho caracteriza-se como uma pesquisa aplicada de natureza
exploratória e experimental, fundamentada nos princípios da Design Science
Research (DSR), metodologia reconhecida por sua capacidade de produzir
artefatos tecnológicos que resolvem problemas concretos enquanto geram
conhecimento científico. A pesquisa foi estruturada em quatro fases
interdependentes e iterativas, alinhadas aos objetivos específicos propostos,
integrando técnicas qualitativas e quantitativas para aplicabilidade prática.

\subsubsection{Fase 1: Investigação de Técnicas de Interface Intuitiva para Usuários Idosos}

A primeira fase concentra-se na compreensão profunda das necessidades e
limitações do público idoso no contexto de interfaces digitais móveis.
Iniciou-se com uma revisão sistemática da literatura abrangendo diretrizes de
acessibilidade para idosos, recomendações de usabilidade específicas para
dispositivos móveis e estudos sobre interação humano-computador com foco nesse
público. Foram consultadas as diretrizes WCAG 2.2 (Web Content Accessibility
Guidelines), adaptadas para o contexto mobile, bem como estudos que investigam
características visuais, cognitivas e motoras associadas ao envelhecimento.

Paralelamente à revisão bibliográfica, serão conduzidas entrevistas
semiestruturadas com potenciais usuários com deficiência e idosos, além de
pessoas que trabalham na área do turismo que atuam com esse público. As
entrevistas seguirão a metodologia de pesquisa participativa, permitindo
identificar preferências de design, barreiras de uso e expectativas
relacionadas a aplicativos turísticos. Os dados coletados deverão ser
organizados através de análise temática, gerando requisitos preliminares sobre
tamanho de fonte, contraste de cores, simplicidade de navegação, feedback
audiovisual e redução de passos para completar tarefas.

\subsubsection{Fase 2: Desenvolvimento das APIs para Integração com o Sistema de Recomendação}

A segunda fase envolverá o projeto e implementação de duas Application
Programming Interfaces (APIs) REST responsáveis pela comunicação entre o
aplicativo móvel, o sistema de recomendação e o banco de dados.

\textbf{API de Recomendação (FastAPI Python):} Implementa o algoritmo d-means de recomendação, integrando dados da Google Maps API como terceiro para selecionar pontos turísticos que combinam perfil de personalidade (BFI) com requisitos de acessibilidade específicos do usuário. Essa camada processa a inteligência central do sistema, retornando lista ordenada de pontos recomendados com score de relevância e metadados de acessibilidade. A arquitetura seguirá princípios RESTful garantindo escalabilidade e manutenibilidade.

\textbf{API de Negócios (Java Spring Boot):} Responsável pela lógica de negócio completa do aplicativo, incluindo autenticação de usuários, processamento do questionário BFI, integração com a FastAPI Python, gerenciamento de dados de acessibilidade, sistema de comentários e avaliações colaborativas, e persistência de dados em PostgreSQL. Implementa requisitos rigorosos de segurança através de autenticação JWT, validação de entrada, CORS configurado, e criptografia de dados sensíveis (AES-256 em repouso, TLS 1.3 em trânsito).

A validação das APIs incluirá testes de carga para avaliar desempenho sob
diferentes volumes de requisições simultâneas, testes de segurança para
identificar vulnerabilidades, testes funcionais para garantir integração
correta entre as camadas, e testes de conformidade com LGPD.

\subsubsection{Fase 3: Implementação do Mínimo Produto Viável (MVP) do Aplicativo Móvel}

A terceira fase consistirá no desenvolvimento do MVP do aplicativo Flutter,
seguindo a metodologia ágil de desenvolvimento em ciclos iterativos curtos que
permitem ajustes contínuos baseados em feedback.

As funcionalidades essenciais implementadas no MVP incluirão: (1) tela de
cadastro e autenticação de usuários com validação segura; (2) questionário
digital baseado no modelo BFI para captura de traços de personalidade; (3)
interface para visualização de pontos turísticos recomendados com indicadores
de acessibilidade (presença de rampas, elevadores, banheiros adaptados,
sinalização tátil); (4) link para mapa interativo mostrando localização dos
pontos; (5) sistema de filtros considerando preferências pessoais e
necessidades de acessibilidade; (6) criação de listas de pontos turísticos
compartilháveis entre usuários; (7) sistema de avaliação colaborativo onde
usuários relatam experiências de acessibilidade e comentam sobre pontos
turísticos.

A interface será projetada considerando os requisitos levantados na Fase 1,
implementando elementos como botões grandes com áreas de toque ampliadas,
fontes ajustáveis (mínimo 16pt), alto contraste entre texto e fundo (4,5:1),
navegação simplificada com poucos níveis hierárquicos, feedback claro para cada
ação do usuário, e suporte a recursos nativos de acessibilidade (VoiceOver no
iOS e TalkBack no Android). A prototipação será realizada utilizando
ferramentas de design como Figma, permitindo validação visual antes da
codificação.

\subsubsection{Fase 4: Avaliação da Eficácia do Aplicativo através de Testes com Usuários}

A fase final foca na validação do aplicativo através de testes de usabilidade e
aceitação com diferentes perfis de usuários. A amostra será selecionada por
conveniência, incluindo: pessoas idosas com diferentes níveis de familiaridade
com tecnologia; pessoas com deficiência física ou mobilidade reduzida; e
profissionais de turismo e acessibilidade que atuam como validadores técnicos.

Os testes serão realizados em ambiente controlado permitindo observação direta
da interação dos usuários com o aplicativo. Tarefas pré-definidas incluem:
realizar cadastro, responder ao questionário de personalidade, visualizar
recomendações, filtrar pontos por critérios de acessibilidade, e observar
comentários de outros usuários. Os resultados serão registrados para análise
posterior da eficácia do sistema.

A avaliação da precisão das recomendações envolverá análise comparativa entre
as sugestões geradas pelo sistema e as preferências reais declaradas pelos
participantes em questionário pós-teste, calculando métricas de acurácia e
relevância. A adequação às condições de acessibilidade será verificada
confrontando as informações apresentadas pelo aplicativo com dados reais dos
estabelecimentos turísticos.

\textbf{Considerações Éticas:} Todos os procedimentos envolvendo participantes humanos serão
conduzidos respeitando os princípios éticos de pesquisa conforme Resolução CNS 466/2012.
Os participantes receberão Termo de Consentimento Livre e Esclarecido (TCLE) detalhando objetivos,
procedimentos, riscos e benefícios, com garantia de confidencialidade e possibilidade de retirada a
qualquer momento sem prejuízos. O projeto será submetido à apreciação do Comitê de Ética em Pesquisa
institucional.

\textbf{Análise de Dados:} Dados quantitativos serão analisados através de estatística descritiva
utilizando ferramentas como R ou Python. Dados qualitativos (transcrições de entrevistas, observaçõ
es dos testes de usabilidade) serão submetidos à análise de conteúdo temática, identificando padrões
recorrentes, dificuldades comuns e sugestões de melhorias, atendendo simultaneamente aos requisitos
técnicos de integração com sistemas de recomendação inteligentes e às necessidades reais de
acessibilidade e usabilidade do público-alvo.
% ---------------------------------------------------------------
% 4 CRONOGRAMA
% ---------------------------------------------------------------
\newpage
\section{Cronograma}
\begin{table}[H]
    \centering
    \begin{tabular}{|p{6cm}|c|c|c|c|c|c|}
        \hline
        \textbf{Atividade}             & \textbf{Nov./25} & \textbf{Dez./25} & \textbf{Jan./26} & \textbf{Fev./26} & \textbf{Mar./26} & \textbf{Abr./26} \\
        \hline
        Revisão de Referencial Teórico & X                & X                &                  &                  &                  &                  \\
        \hline
        Modelagem do sistema           &                  & X                &                  &                  &                  &                  \\
        \hline
        Criação de endpoints da API    &                  &                  & X                &                  &                  &                  \\
        \hline
        Criação de Protótipo           &                  &                  &                  & X                &                  &                  \\
        \hline
        Criação de telas do App        &                  &                  &                  & X                & X                &                  \\
        \hline
        Integração do App com a API    &                  &                  &                  & X                & X                &                  \\
        \hline
        Testes e validação             &                  &                  &                  &                  & X                & X                \\
        \hline
    \end{tabular}
    \caption{Cronograma de execução do projeto.}
\end{table}

% ---------------------------------------------------------------
% 5 REFERÊNCIAS
% ---------------------------------------------------------------
\newpage
\section{Referências}
\begin{singlespace}
    \setlength\bibitemsep{10pt}
    \sloppy
    %\nocite{*}
    \printbibliography[heading=none]
\end{singlespace}
\end{document}

